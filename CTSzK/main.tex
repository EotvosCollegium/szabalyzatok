\documentclass{rulebook}


\begin{document}
\section*{ELTE Eötvös József Collegium \\ \vspace{0.5em} Oktatási, Tanulmányi Szabályzat és Követelményrendszer} 

\vspace{2em}


\section*{\normalfont Preambulum} 

Az Eötvös József Collegium Oktatási, Tanulmányi Szabályzata és Követelményrendszere (a továbbiakban CTSzK) az Eötvös József Collegiumban folyó oktatási és tudományos munka kereteit meghatározó és követelményeit szabályozó dokumentum, amely minden bejáró és bentlakó tagra egyaránt, megszorítások nélkül kötelező. Az ELTE Szervezeti és Működési Szabályzatának (továbbiakban ELTE SzMSz) II. kötetéhez, a Hallgatói Követelményrendszerhez (a továbbiakban HKR) képest – annak szellemében – a CTSzK további megkötéseket tartalmaz, amelyek az Eötvös Collegiumban folyó tehetséggondozás minőségbiztosításáért felelnek. Eötvös-collegistának lenni a több mint százéves hagyomány folytatását jelenti, amely ennek megfelelően nemcsak tudományos, hanem kulturális kötelezettségekkel is jár, többek között a társadalmi felelősségvállalás, a közösségiség és a nemzetközi színvonalra való törekvés tekintetében.


\section{A szabályzat hatálya, a collegista kötelességei}

\emph{HKR 25. §  ,,Az Egyetem a kiemelkedő képességű hallgatók tehetségének kibontakoztatását szervezett keretek között {\normalfont [\dots]} a szakkollégiumi képzés formájában {\normalfont [\dots]} segíti elő.''}

\begin{enumerate}
	\item	Jelen CTSzK hatálya kiterjed az ELTE Eötvös József Collegium minden tagjára (a továbbiakban collegista) a collegista státusz létrejöttétől annak megszűntéig.
	\item	A collegista kötelessége, hogy a szakkollégiumi tanulmányait Eötvös-collegistához méltó módon, felelősséggel teljesítse.
	\item	A collegistának szakkollégiumi tanulmányait, mely magában foglalja a nyelvtanulásra vonatkozó előírásokat, a műhelye(i) által megszabott tanulmányi követelményeket, továbbá az osztatlan tanárképzésben részt vevő collegisták számára az Eötvös Collegium-i Tanárképzési Munkacsoport (a továbbiakban ECTKM) előírásait, a CTSzK-ban meghatározott módon kell teljesítenie.
\end{enumerate}


\section{A collegiumi felvételi rendje}

\emph{HKR 27. §  (2)  ,,A szakkollégista státusz sikeres szakkollégiumi felvételivel szerezhető meg, melynek rendjéről az egyes szakkollégiumok felvételi szabályzata rendelkezik.''}

\begin{enumerate}
	\item	A Collegiumba felvételt, azaz collegista (szakkollégiumi) státuszt kizárólag sikeres felvételi eljárás során lehet nyerni (ez alól kivételt képez az 5. § (8)-ban említett eset).
	\item	A Collegiumba felvételizni BA/BSc-képzés első, második vagy harmadik évének, MA/MSc-képzés első évének, illetve osztatlan tanárképzés első, második, harmadik vagy negyedik évének megkezdése előtt lehetséges.
	\item	A Collegiumba az ELTE BTK-ra, IK-ra, TáTK-ra és TTK-ra felvett nappali tagozatos hallgatók felvételizhetnek, illetve kivételes esetben olyanok, akik a Collegium valamelyik műhelyéhez kapcsolódó minort végeznek.
	\item	A felvételiző egyik szakjának, vagy kivételes esetben egyik minorjának minden esetben az ELTE olyan egyetemi szakjának kell lennie, amelynek megfelelő szakterületen a Collegiumban műhely működik (a szakok és minorok pontos besorolásáról az egyes műhelyek szabályzatai rendelkeznek).
	\item	A felvételire elektronikus formában lehet jelentkezni az adott év collegiumi felvételi tájékoztatójában meghatározott időpontjáig.
	\item	A felvételire való jelentkezésnek minden esetben tartalmaznia kell:
	\begin{enumerate}
		\item	a kitöltött felvételi űrlapot,
		\item	a jelentkező önéletrajzát,
		\item	az Oktatási Hivatal által kiállított felvételi határozat, vagy kivételes esetben az ELTE által kibocsátott szakfelvételi engedély fénymásolatát, vagy a szakirányfelelős írásbeli nyilatkozatát a minor felvételéről.
	\end{enumerate}
	\item	A felvételire való jelentkezéshez csatolható továbbá:
	\begin{enumerate}
		\item	az egyetemi szakjához kapcsolódó, és általa legjobbnak ítélt írásbeli dolgozatának egy példánya,
		\item	a szakjának megfelelő, vagy kivételes esetben más, őt jól ismerő szaktanár ajánlása,
		\item	az űrlapon feltüntetett eredményeit, tevékenységeit igazoló dokumentumok fénymásolata.
	\end{enumerate}
	\item	A collegiumi felvételi keretszámról a collegiumi férőhelyek ismeretében az igazgató dönt, a tanári kar, a fejkopogtató bizottság és a felvételi diákbizottságok javaslatait figyelembe véve.
	\item	Kívánatos, hogy a bentlakó hallgatók összlétszáma ne haladja meg a 120 főt, hogy a bejáró collegisták létszáma ne haladja meg a bentlakó collegisták számát, valamint, hogy a felvett collegisták között a TTK-s és IK-s hallgatók összlétszáma ne süllyedjen a BTK-s és TáTK-s hallgatók összlétszámának fele alá.
	\item	A felvételi vizsga négy fordulós:
	\begin{enumerate}
		\item	szakmai bizottság meghallgatása,
		\item	diákbizottság meghallgatása,
		\item	igazgatói beszélgetés,
		\item	igazgatói javaslat alapján a fejkopogtató bizottság meghallgatása.
	\end{enumerate}
	Bejáró státuszra jelentkezőkre csak az első kettő pont vonatkozik.
	\item	A felvételi anyagban lévő adatok adatvédelmi szempontból érzékenynek minősülnek, így azok megtekintésével minden fél tudomásul veszi, hogy magára nézve az általános adatvédelmi elveket kötelezőnek tekinti, és az adatokat sem közvetlenül, sem közvetve harmadik személynek (ebbe beleértendő az adott hallgatóra vonatkozó felvételi eljárásban részt nem vevő collegista is) nem adja ki. Ez alól egyedüli kivételt a felvett hallgatók e-mail címei jelentenek, amelyeket kérésére a választmányi elnök rendelkezésére kell bocsátani.
	\item	A szakmai bizottság munkáját a műhelyvezető, vagy akadályoztatása esetén egy általa megbízott szaktanár vezeti. A szakmai zsűri összetételét és működését a HKR és a CTSzK szellemében az egyes műhelyek szabályzata tartalmazza. %TODO milyen HKR? pontos hivatkozás
	\item	A diákbizottságok munkáját a Választmány a műhelytitkárokkal egyeztetve koordinálja.
	\item	A harmadik fordulóra (igazgatói beszélgetés) javasolt jelentkezők listáját a diákbizottság javaslatait figyelembe véve a szakmai bizottság állítja össze.
	\item	Az igazgatói beszélgetés után a felvételizők az igazgató javaslatára továbblépnek a felvételi eljárás negyedik fordulójába.
	\begin{enumerate}
		\item	Kiemelkedően jól teljesítő hallgató esetében, amennyiben őt a szakmai bizottság és a diákbizottság is a legjobbak közé tartozónak nyilvánította, az igazgató közvetlenül javasolhatja a tanári értekezletnek a jelentkező felvételét, és el lehet tekinteni a felvételi eljárás negyedik fordulójától.
	\end{enumerate}
	\item	A fejkopogtatást (a negyedik fordulót) a fejkopogtató bizottság végzi, amely az igazgatóból, az aligazgatóból, a kurátorból, a választmányi elnökből, valamint az Igazgató által delegált, a Collegiumhoz kötődő egykori collegistákból áll össze.
	\item	A felvételi eljárás eredményét a Collegium Tanári Kara (a továbbiakban Tanári Kar) felvételi értekezlete (továbbiakban: felvételi tanári értekezlet) értékeli, amelyen a Tanári Karon kívül tanácskozási és javaslattételi joggal jelen vannak a szakmai bizottságok elnökei, a választmányi elnök és a diákbizottságok egy-egy tagja.
	\item	A felvettek névsoráról a felvételi tanári értekezlet javaslatának figyelembe vételével az igazgató dönt. A felvételi tanári értekezlet a kiváló, ám helyhiány miatt bentlakó státuszra fel nem vehető hallgatókból várólistát állíthat össze, akik bejáró collegista státuszt kapnak, és rangsorolásuk alapján collegiumi férőhely megüresedése esetén soron kívül bentlakó hallgatókká válhatnak. A várólista alapján legkésőbb a tavaszi félév szorgalmi időszakának első napján lehet bentlakó státuszt kapni. Az értekezletről jegyzőkönyv készül, amelyet az igazgató a választmányi elnök és legalább egy műhelyvezető az aláírásával hitelesít.
	\item	A döntésről a felvételi tanári értekezletet követő öt munkanapon belül az igazgató elektronikus úton értesíti a pályázókat. A döntés ellen fellebbezésnek helye nincs.
	\item	A BA/BSc-képzésben felvett hallgatók collegiumi tagsága, amennyiben a collegista teljesíti a jelen szabályzatban foglalt követelményeket, hat félévig, azaz a BA/BSc-képzés végéig szól.
	\begin{enumerate}
		\item	Amennyiben a collegista a BA/BSc-képzést hat félév alatt elvégzi, és a műhelyének megfelelő MA/MSc-képzésben folytatja tanulmányait az ELTE-n, a collegiumi jogviszonya a collegista által leadott kérvény és az illetékes műhelyvezető ajánlása alapján további négy félévvel meghosszabbodik. %TODO ne kelljen kérvény
		\item	Amennyiben a collegista a BA/BSc-képzést hat félév alatt elvégzi, de tanulmányait nem a korábban végzett szakjának megfelelő, az ELTE olyan egyetemi MA/MSc-képzésén folytatja, amelynek megfelelő szakterületen a Collegiumban műhely működik, kérvényezheti az átvételét az új szakjának megfelelő műhelyhez. Kérvényéről a korábbi és az új műhelyvezető együttes véleménye, illetve javaslata alapján, a tanári értekezlet állásfoglalásának figyelembe vételével az igazgató dönt.
		\item	Amennyiben a collegista egyedi méltányossági kérelem alapján nyolc félév alatt elvégezte a BA/BSc-képzést, kérvényezheti, hogy tagságát az MA/MSc-képzés idejére további négy félévvel hosszabbítsák meg. Kérvényéről a műhelyvezető javaslatára a tanári értekezlet állásfoglalását figyelembe véve az igazgató dönt. %TODO mindkét kérvényt átnézni
		\item	Amennyiben a collegista egyszerre két, vagy annál több BA/BSc-szakot végez, kérelmezheti, hogy a collegiumi tagsága a másodszor megkezdett BA/BSc-képzésének végéig, azaz további két félévvel hosszabbodjon meg. Ebben az esetben, amennyiben a műhelyének megfelelő MA/MSc-képzésben folytatja tanulmányait az ELTE-n, a collegiumi jogviszonya az illetékes műhelyvezető ajánlása alapján további négy félévvel meghosszabbodik.
	\end{enumerate}
	\item	Az MA/MSc-képzésben felvett hallgatók collegiumi tagsága, amennyiben a collegista teljesíti a jelen szabályzatban foglalt követelményeket, négy félévre szól.
	\begin{enumerate}
		\item	Amennyiben a collegista egyszerre két, vagy annál több MA/MSc-szakot végez, kérelmezheti, hogy collegiumi tagsága a másodszor megkezdett MA/MSc-képzésének végéig, azaz további két félévvel hosszabbodjon meg. %TODO csak ilyenkor?
	\end{enumerate}
	\item	Az osztatlan tanárképzésben felvételt nyert hallgatók collegiumi tagsága, amennyiben a collegista teljesíti a jelen szabályzatban foglalt követelményeket, a tanterv által ajánlott képzési idő végéig szól.
	\begin{enumerate}
		\item	Amennyiben a collegista ezen félévek alatt nem végezte el a képzést, kérvényezheti, hogy tagságát további két félévvel hosszabbítsák meg. Kérvényéről a műhelyvezető javaslatára a tanári értekezlet állásfoglalását figyelembe véve az igazgató dönt.
	\end{enumerate}
	\item	A felvételi eljárásban részt vevőhallgatók elszállásolásáról, valamint a beköltözés menetéről a CTSzK mellékletét képező collegiumi Házirend rendelkezik.
\end{enumerate}


\section{Seniorok felvétele}

\begin{enumerate}
	\item	Azok a collegisták és volt collegisták, akik valamely egyetem akkreditált doktori iskolájába igazoltan bejelentkeznek vagy ilyen doktori iskola képzésében részt vesznek, jogosultak bentlakó vagy bejáró seniori pályázat benyújtására. A felvehető bentlakó seniorokra vonatkozó irányszám évente kettő.
	\item	Mind a bejáró, mind a bentlakó seniori státuszra pályázatot kell benyújtani, amelynek határidejét az igazgató határozza meg, és amelyet legkésőbb a beadási határidő előtt tíz munkanappal az érintettekkel az igazgatói iroda előtti hirdetőtáblán tudatni kell.
	\item	A pályázatnak tartalmaznia kell:
	\begin{enumerate}
		\item	a pályázó szakmai önéletrajzát és publikációs listáját,
		\item	a doktori kutatási tervet,
		\item	a pályázó terveit a Collegium oktatói és kutatói munkájában történő részvételről,
		\item	az illetékes műhelyvezető és a témavezető ajánlását.
	\end{enumerate}
	\item	A seniori pályázat benyújtásának feltétele legalább két nyelvből középfokú C-típusú államilag elismert nyelvvizsga (B2-es szint) megléte. %TODO C-típusú?
	\item	A pályázatokat a műhelyvezetőnek a műhelytitkárral közösen kialakított véleményének és a Választmány állásfoglalásának ismeretében a Tanári Kar értekezletén értékeli és rangsorolja. A bentlakó és bejáró seniorok felvételéről a Tanári Kar rangsorát figyelembe véve az igazgató dönt.  %TODO a Választmány? Műhelytitkár?
	\item	A doktori iskolába történt felvétel igazolása után válik véglegessé a felvettek névsora.
	\item	A seniori státusz egy évre szól. A senioroknak minden félévben beszámolót kell benyújtaniuk az adott féléves teljesítményükről, amelyet a Tanári Kar értekezletén értékel. A Tanári Kar véleményét figyelembe véve a seniori státusz egy évvel történő meghosszabbításáról egy hallgató esetén legfeljebb két alkalommal az igazgató dönt.
\end{enumerate}


\section{A Collegium tanulmányi rendje}

\emph{HKR 27. § (1) ,,A szakkollégiumi képzési formára befogadást nyert egyetemi hallgatók -- továbbiakban: a szakkollégista hallgatók -- tanulmányaik egy részét a párhuzamosan meghirdetett kurzusok közül a szakkollégista hallgatóknak hirdetett kurzusok elvégzésével teljesíthetik.''}

\begin{enumerate}
	\item	A Collegium tanulmányi rendje a bejáró és bentlakó BA/BSc-képzésben, osztatlan tanárképzésben és MA/MSc-képzésben részt vevő collegistákra vonatkozik. A seniorok a tanulmányi kötelezettségeiket a doktori iskolában teljesítik.
	\item	A Collegium tagjai a végzettségüket igazoló oklevelet az ELTE BTK-n, IK-n, TáTK-n, TTK-n, vagy kivételes esetben a főszakjukhoz kapcsolódó egyéb karon szerzik meg. A minőségbiztosítási szempontoknak megfelelt, akkreditált műhelyprogramok a három-, öt-, hat- illetve kétéves képzés végén – a kari sajátosságoknak megfelelően – diplomabetétlapot, vagy betétlapnak nem minősülő oklevelet bocsátanak ki (az akkreditált műhelyprogramokról részletesen az 5. § rendelkezik). %TODO ez mi?
	\item	A collegiumi tehetséggondozás formája általában kiscsoportos, szemináriumi jellegű, intenzív szaktárgyi oktatás, szaktárgyanként heti 2-6 órában, de emellett más módszerek (mentori foglalkozás, előadás, szakosított programok stb.) is alkalmazhatók. Kívánatos, hogy a Collegium nyújtotta tehetséggondozási formát kiegészítsék a kari tehetséggondozási formák (tutorálás, honorácior státusz, Tudományos Diákköri Konferencia), amelyek önmagában a Collegium által nem biztosítható módokon segítik a tehetségek kibontakozását.
\end{enumerate}

	\emph{HKR 27. § (3) ,,Szakkollégiumi kurzusra csak szakkollégista hallgatók iratkozhatnak be, de az oktató az általa vezetett szakkollégiumi kurzusra kivételesen – egyéni elbírálás alapján – az ilyen képzésben részt nem vevő hallgatók felvételét is engedélyezheti.''}
		
	\emph{HKR 55. § (5) ,,A szakkollégiumi kurzust a meghirdetésétől függően nem szakkollégisták is	fölvehetik, ha a szakkollégisták nem töltik ki a meghirdetett maximális létszámot. A szakkollégisták által fölvett szakkollégiumi kurzussal teljesített tanegységek nem számítanak	bele az Nftv. 49. § (2) bekezdésben foglalt 10 \%-nyi, külön költségtérítés/önköltség és térítési díj fizetése nélküli keretbe. {\normalfont [\dots]} ''}
	
	\emph{HKR 27.  §  (5) ,,Szakkollégiumi kurzus hirdethető nemszakos (általános bölcsészképző, „közismereti”) és szakos tárgyakból egyaránt.''}
	
	\emph{HKR 27.  §  (6) ,,A szakkollégiumi kurzusok indításáról a szakkollégium igazgatója dönt.''}
	
	\emph{HKR 55. §  (4)  ,,Szakkollégiumi kurzust -- ha a kurzus valamely szak (ideértve a felsőoktatási	szakképzést is) tantervében szereplő tanegység teljesítésére alkalmas (kreditszerző) -- a szakkollégium igazgatója a dékán jóváhagyásával hirdet meg.''}
	
\begin{enumerate}[resume]
	\item	A Collegium hallgatójának be kell jelentenie az igazgatónak, ha félévet halaszt. 
	\item	Az a collegista, aki collegiumi tagságáról le kíván mondani, azt a titkárság által kiküldött formanyomtatványon köteles jelezni.
	\item	A collegiumi tagság feltétele a műhelytagság az ELTE SzMSz I. 4/g melléklete (továbbiakban:  SzMSz) 40--48. §-ban meghatározott szakmai műhelyben).
	\item	A Collegiumban az alábbi szakmai műhelyek működnek:
	\begin{enumerate}
		\item	Angol-Amerikai Műhely
		\item	Aurélien Sauvageot Francia Műhely
		\item	Biológia-Kémia Műhely
		\item	Bollók János Klasszika-Filológia Műhely
		\item	Filozófia Műhely
		\item	Germanisztika Műhely
		\item	Informatikai Műhely
		\item	Magyar Műhely
		\item	Matematika-Fizika Műhely
		\item	Mendöl Tibor Földrajz-Földtudomány-Környezettudomány Műhely
		\item	Olasz Műhely
		\item	Orientalisztika Műhely 
		\item	Skandinavisztika Műhely 
		\item	Spanyol Műhely
		\item	Szlavisztika Műhely
		\item	Társadalomtudományi Műhely
		\item	Történész Műhely
	\end{enumerate}
	\item	Az egyes műhelyekben folyó szakmai munka formáját és tanulmányi követelményeket a műhelyszabályzatok rögzítik, amelyek a HKR-hez és a CTSzK-hoz képest további követelményeket támaszthatnak. A műhelyszabályzatokat a műhelygyűlés javaslatára az igazgató fogadja el. A műhelyszabályzatok jelen szabályzat mellékletét képezik. %TODO melléklet
\end{enumerate}


\section{Műhelyprogramok és akkreditációjuk}

\begin{enumerate}
	\item	Kívánatos, hogy a szakmai műhelyek modul jellegű, az egyetemi képzést nem kiváltó, hanem kiegészítő műhelyprogramokat – curriculumokat -- (továbbiakban: programok) indítsanak.
	\item	A programokat a Tanári Kar értekezlete akkreditálhatja. A programokért felelős műhelyek csak felmenő rendszerben (azaz a teljes program elvégzése után) bocsáthat ki oklevelet. Az akkreditáció a szakkollégiumi munka belső minőségbiztosítása.
	\item	Az akkreditált programokban minden műhelytagnak részt kell vennie, azonban egy collegistának egynél több programban nem kell részt vennie.
	\item	Külön programot kell kidolgozni a BA/BSc-képzésben, osztatlan tanárképzésben és az MA/MSc-képzésben részt vevő hallgatók számára, és ezeket külön kell akkreditáltatni, azonban kívánatos, hogy a két program valamilyen módon egymásra épüljön.
	\item	A program akkreditálásának minimális feltételei:
	\begin{enumerate}
		\item	előre meghatározott tanegységek meghatározott oktatói gárdával és programvezetővel,
		\item	előre meghatározott követelményrendszer és tanrend,
		\item	Félévente legalább két különböző kurzus meghirdetése, amelyeket két különböző, tudományos fokozattal rendelkező oktató tart,
		\item	folyamatos tutorálás biztosítása állandó tutori körrel,
		\item	a kurzusokon kívül legalább egy folyamatos szakmai tevékenység (pl.: szakfordítás, TDK-dolgozat írás stb.),
		\item	meghatározott cél megjelölése (pl.: egy kompetencia fejlesztése, a szaknál szűkebb területről adott bővített képzés stb.).
	\end{enumerate}
	\item	Az akkreditált programot elvégzett hallgató számára a program elvégzését a 4.  § (2)-ben foglaltaknak megfelelően igazolni kell, a szakmai műhely, a program és a programvezető megnevezésével.
	\item	Akkreditált program akkor veszíti el akkreditációját, amennyiben két egymást követő félévben nem sikerül teljesíteni az 5. § (4) c) és d)-ben foglaltakat, ekkor a programot újra kell akkreditálni.
	\item	Kivételes esetekben a Tanári Kar véleményét figyelembe véve az igazgató dönthet a műhelyek munkájához nem kapcsolódó, vagy több műhely, szak területét érintő programok integrálásáról. Ebben az esetben az adott program tagjai az igazgató javaslatára egyéni mérlegelés alapján bejáró collegista státuszt kaphatnak, és szervezeti egységük alkalmi kutatóműhelyként (vö. SzMSz 43. §) a collegiumi képzés részévé válik. 
\end{enumerate}


\section{A nyelvtanulásra vonatkozó követelmények}

\begin{enumerate}
	\item	Minden BSc/BA-képzés és osztatlan tanárképzés első három évében felvett collegista kötelezően választ az angol, francia, német, olasz, latin, spanyol és ógörög nyelvek közül, melynek tanulását a CTSzK mellékletét képező ALFONSÓ program szabályozásának megfelelően köteles teljesítenie.
	\item	A BTK-s és TáTK-s hallgatóknak az első collegiumi évükben két egymást követő félévben latin nyelvből sikeres vizsgát kell tenniük (azaz az ötfokozatú skálán legalább elégséges, a háromfokozatú skálán legalább megfelelt minősítést szerezniük). Kívánatos, hogy ez irányú kötelezettségüket a Collegiumban meghirdetett kurzus keretében teljesítsék. A Filozófia Műhely hallgatói választhatnak a latin és az ógörög nyelv között. %TODO kell az egyértelműsítés
\end{enumerate}



\section{Általános követelmények és szankciók}


\begin{enumerate}
	\item	Az általános, a nyelvtanulásra vonatkozó előírásokat, valamint a műhelyek tanulmányi követelményeinek teljesülését a Tanári Kar félév végi értekezlete értékeli.
	\item	A Tanári Kar értekezlete:
	\begin{enumerate}
		\item	meghallgatja a műhelyvezetők beszámolóját,
		\item	meghallgatja a nyelvtanárok beszámolóját,
		\item	javaslatot tesz a dicséretben, illetve figyelmeztetésben vagy elmarasztalásban részesítendő hallgatók személyére,
		\item	javaslatot tesz a követelményeket nem teljesítő hallgatók kizárására,
		\item	javaslatot tesz a méltányossági kérelmek elbírálására;
		\item	valamint, ha lehetősége van rá, a Collegium minden tagjának szakmai tevékenységét egyenként értékeli.
	\end{enumerate}
	\item	A collegisták által a tanulmányi előmenetelükről beszolgáltatott adatokat az igazgató megbízására a titkárság összegyűjtheti, tárolhatja és rendszerezheti.  Ezek az adatok nem nyilvánosak, adatvédelmi szempontból érzékenynek számítanak, és a titkárság a tárolásukkal magára nézve kötelezőnek ismeri el a vonatkozó adatvédelmi irányelveket.  Ezek az adatok nevesítés nélkül egyesével vagy összességükben felhasználhatóak a Collegium eredményességét demonstráló statisztikák elkészítésére, azonban csak az illető collegisták beleegyezésével nevesíthetők.
	\item	A collegiumi tagság automatikusan megszűnik, olyan collegista esetében:
	\begin{enumerate}
		\item	a hallgató egy aktív félévben egyetlen collegiumi órát sem vett fel (ez alól kivételt képeznek azon collegisták, akik műhelye vagy műhelyei egyetlen órát sem hirdettek meg, és nyelvtanulási kötelezettségüket már teljesítették, illetve az ösztöndíjjal külföldön tartózkodó hallgatók), 
		\item	a hallgatónak a tanulmányi átlaga két egymást követő félévben 4,25 alá süllyed
		\begin{enumerate}
			\item	ahol a hagyományos átlagszámítás az érvényes, melybe minden szöveges értékelésű és nullkredites tárgy is beleszámít, illetve a BTK-s és TáTK-s kezelési körben meghirdetett kurzusok esetében az elhagyott tanegység értéke nulla,
			\item	a hallgató mentesül a 7. § (4) b. rendelkezés alól, amennyiben a műhelyvezető támogatásával a hallgató kérelmezésére kezdeményezett vizsgálat alapján teljesítménye mindkét kérdéses félévben az adott szakon azonos számú aktív félévvel rendelkező hallgatók kreditindexe alapján felállított lista legjobb 10\%-ához tartozik,
		\end{enumerate}
		\item	a hallgató a mintatanterv szerint a félévének megfelelő teljesített kreditszámától több mint 33 kredittel elmarad,
		\item	a hallgató műhelytagságát elveszítette, és másik műhelynek nem tagja.
	\end{enumerate}
	\item	A Collegiumból elbocsátható, aki:
	\begin{enumerate}
		\item	két egymást követő félévben nem tesz eleget a collegiumi foglalkozások szakmai követelményeinek,
		\item	aki Collegiumban felvett óráját nem teljesítette,
		\item	a hallgató nem teljesíti a Collegium 6. §-ban leírt nyelvtanulásra vonatkozó követelményeket,
		\item	a hallgató a CHÖK két egymást követő Közgyűlésétől igazolatlanul távol marad.
	\end{enumerate}
	\item	Az (5) vonatkozásában a műhelyvezető véleménye és értékelése a mérvadó.
	\item	A collegiumi tagság megszüntetéséről a Tanári Kar, a műhelyvezető és a választmányi elnök javaslatára az igazgató dönt.
	\item	A tanulmányi okból történő kizárás nem fegyelmi büntetés, ellene fellebbezésnek helye nincs, kivéve, ha a kizáró határozatra adminisztratív hiba miatt került sor.
\end{enumerate}


\section{Méltányossági kérelem}

\begin{enumerate}
	\item	A 2. § (20)--(22) és a 7. § (4)--(5) vonatkozó rendelkezéseivel kapcsolatban az igazgatóhoz írásban méltányossági kérelem nyújtható be, legkésőbb a Tanári Kar tárgyfélévi értekezlete előtti ötödik munkanapon.
	\item	Ugyaneddig méltányossági kérelmet nyújthat be az a bejáró collegista, aki státuszát bentlakóra szeretni módosítani.
	\item	A méltányossági kérelem alapja rendkívüli tanulmányi, egészségügyi, anyagi vagy családi indok lehet, amelyet részletesen kifejteni és indokolni kell. A méltányossági kérelemhez szakmai önéletrajzot is mellékelni kell.
	\item	A méltányossági kérelmeket a Tanári Kar értekezlete javaslatára a műhelyvezető és a választmányi elnök véleményét figyelembe véve az igazgató bírálja el.
	\item	A méltányossági kérelemről szóló döntésről legkésőbb a Tanári Kar értekezlete utáni ötödik munkanapon írásos és elektronikus formában tájékoztatni kell az érintetteket.
	\item	A méltányossági kérelemről szóló döntés ellen fellebbezésnek helye nincs, kivéve, ha adminisztratív hiba történt.
\end{enumerate}	


\section{Záró, átmeneti és értelmező rendelkezések}

\begin{enumerate}
	\item	A jelen CTSzK-ban nem szabályozott kérdésekben az SzMSz és a HKR az irányadó.
	\item	Jelen CTSzK az SzMSz első számú melléklete.
	\item	Jelen CTSzK 2018. február 22-tól érvényes.
	\item	Jelen szabályzat elfogadásáról és jóváhagyásáról az SzMSz 58. § (2)-(3) pontja rendelkezik. %TODO nem rendelkezik
	\item	Jelen CTSzK mellékletét képezi az ALFONSÓ program és a Házirend.
\end{enumerate}

\end{document}