\documentclass{../styles/rulebook}

\begin{document}
\section*{ELTE Eötvös József Collegium \\ \vspace{0.5em} Hallgatói Önkormányzat -- Alapszabály}

\vspace{2em}


\section*{\normalfont Preambulum} 

Az Eötvös Loránd Tudományegyetem (továbbiakban ELTE) Eötvös József Collegiumának (továbbiakban Collegium) Közgyűlése a Szenátustól az ELTE Szervezeti és Működési Szabályzatában rögzített felhatalmazása és kötelezettsége alapján a Collegiumi Hallgatói Önkormányzat (továbbiakban CHÖK) alapszabályát az alábbiakban állapítja meg.


\section{Általános rendelkezések}
\begin{enumerate}
	\item Az Önkormányzat 
	\begin{enumerate}
		\item neve: Collegium Hallgatói Önkormányzata, 
		\item székhelye: 1118 Budapest, Ménesi út 11--13.,
		\item nemzetközi neve: Students’ Council of Eötvös József Collegium.
	\end{enumerate}
	\item A CHÖK a Collegium hallgatóinak a felsőoktatásról szóló, 2011. évi CCIV. törvény alapján létrehozott, demokratikus elven működő érdekképviseleti szervezete, amely kizárólagosan gyakorolja a hallgatói jogviszonyból eredő kollektív hallgatói jogokat, illetve a jogszabályokban és az collegiumi szabályzatokban ráruházott hatásköröket. 
	\item A CHÖK tagja a Collegium minden bentlakó és bejáró aktív jogviszonyú hallgatója, valamint bentlakó és bejáró seniorja.
	\item A CHÖK egyéb, nem kari szervezetbe tartozó szervezeti egység. Nem jogi személy, azonban megilletik a jogszabályban, az egyetemi szabályzatokban és a Collegium szabályzataiban megállapított jogok, és terhelik a kötelezettségek.
	\item A CHÖK a Collegium érdekképviseletét és érdekvédelmét látja el, dönt a jelen Szabályzatban ráruházott személyi és egyéb kérdésekben, gyakorolja az ELTE és a Collegium Szervezeti és Működési Szabályzata (ELTE Szervezeti és Működési Szabályzat I. 4/g. melléklete, továbbiakban: SzMSz) által a CHÖK-re ruházott döntési, javaslattételi, véleményező és ellenőrző jogköröket, részt vesz a hallgatók ügyeinek intézésében és mindezeken kívül az Önkormányzat céljaival összeegyeztethető egyéb tevékenységet folytat. (vö. SzMSz 39. §)
	\item A CHÖK tisztségviselőit és képviselőit a Collegium hallgatói választják meg jelen Alapszabályban meghatározott módon. Ennek során minden aktív státuszú hallgató választó, illetve választható. A választójog általános, egyenlő és titkos.
	\item A CHÖK tagjai közül tisztségviselőnek minősülnek a Választmány Elnöke, és Alelnökei.
	
\end{enumerate}


\section{A CHÖK feladatai}

\begin{enumerate}
	\item A CHÖK a felsőoktatásról szóló törvényben és egyéb jogszabályokban, valamint az Egyetem szabályzataiban meghatározott döntési, véleményezési, egyetértési, javaslattevő, ellenőrzési és delegálási jogokat, illetve az azokban ráruházott egyéb hatásköröket gyakorolja.
	\item A CHÖK feladatai:
	\begin{enumerate}
		\item A kari és egyetemi hallgatói önkormányzatokkal együttműködve ellátja a Collegium hallgatóinak érdekképviseletét valamennyi őket érintő kérdésben,
		\item együttműködik az országos és egyetemi szakkollégiumi hallgatói szervezetekkel,
		\item egyéb külügyekben képviseli a Collegium hallgatóinak érdekeit,
		\item részt vesz a felvételi bizottságok munkájában, (vö. a Collegium Oktatási, Tanulmányi Szabályzata és Követelményrendszere, továbbiakban CTSZK 2§ 11-14.) a tudományos és szakmai diákkörök szervezésében;
		\item kialakítja saját szervezeti és működési rendjét, és dönt a jelen Alapszabályban és az SzMSz-ben ráruházott személyi és egyéb kérdésekben;
		\item intézi és segíti a hallgatók Collegiummal kapcsolatos ügyeit, valamint segíti a Collegiumon kívüli tevékenységüket,
		\item folyamatosan tájékoztatja a hallgatókat a CHÖK tevékenységéről, a Collegium életével kapcsolatos kérdésekről,
		\item támogatja a hallgatók szakmai és egyéb közösségi tevékenységét, a hallgatói kezdeményezéseket képviseli és megvalósítja.
	\end{enumerate}
	\item A CHÖK egyetértést gyakorol:
	\begin{enumerate}
		\item az SzMSz elfogadásakor, illetve módosításakor;
		\item a Collegiumban működő jóléti, kulturális, sport célú ingatlanok és intézmények, szervezeti egységek, helyiségek rendeltetésszerű használatának megváltoztatásakor, megszüntetésekor és hasznosításával kapcsolatban;
		\item a hallgatói célú pénzeszközök felhasználásában;
		\item a hallgatói fegyelmi és kártérítési szabályzat elfogadásakor, módosításakor;
		\item a tanulmányokhoz nem kapcsolódó szolgáltatási díjak megállapításakor;
		\item a Collegiumra vonatkozó szabályzatok elfogadásakor és módosításakor.
	\end{enumerate}
\end{enumerate}


\section{A CHÖK felépítése}

\begin{enumerate}
	\item A CHÖK tevékenységét a következő szervek irányítják és ellenőrzik:
	\begin{enumerate}
		\item A CHÖK Közgyűlése (továbbiakban: Közgyűlés),
		\item A CHÖK Közgyűlése által választott Választmány,
		\item A CHÖK Közgyűlése által választott kuratóriumi diáktagok.
	\end{enumerate}	
\end{enumerate}


\section{A Közgyűlés}

\begin{enumerate}
	\item A CHÖK legfontosabb döntéshozó szerve a Közgyűlés. A Közgyűlés a CHÖK-öt érintő valamennyi ügyben dönthet, és valamennyi alsóbb szintű döntést felülbírálhat.
	\item A Közgyűlés kizárólagos hatáskörébe tartozik:
	\begin{enumerate}
		\item a Választmány megválasztása, beszámoltatása, felmentése,
		\item a CHÖK Alapszabályának megalkotása, módosítása,
		\item az SZMSZ, a CTSZK és a Collegium Házirendjének módosításával kapcsolatos javaslattételi és véleményezési jogának gyakorlása (vö. SzMSz 38. § (a)),
		\item a collegiumi igazgató kinevezésének és felmentésének véleményezése,
		\item a Collegium fejlesztési tervének véleményezése,
		\item a Kuratórium diáktagjainak választása (vö. SzMSz 15. §),
		\item valamint döntéshozatal minden olyan ügyben, amelyet jelen Szabályzat, valamint bármely egyetemi szabályzat a Közgyűlés kizárólagos hatáskörébe utal.
	\end{enumerate}
\end{enumerate}


\section{A Közgyűlés tagjai}

\begin{enumerate}
	\item A Közgyűlés tagja a CHÖK valamennyi tagja.
	\item Tanácskozási joggal az választmányi elnök által meghívott személyek. Javasolt meghívni az ELTE BTK, IK, TÁTK és TTK hallgatói önkormányzatának képviselőit, az ELTE EHÖK és KolHÖK elnökét, a Baráti Kör elnökét, a Tanári Kar tagjait és a Collegium igazgatóját.
\end{enumerate}

\section{A Közgyűlés működése}

\begin{enumerate}
	\item A Közgyűlés üléseit a választmányi elnök, akadályoztatása esetén a választmányi alelnökök hívhatják össze és vezetik le.
	\item A Közgyűlést a választmányi elnök nyitja meg és vezeti le.
	\item A Közgyűlést szemeszterenként legalább egyszer a választmányi elnök köteles összehívni.
	\item A Közgyűlést 20 collegistának vagy a Collegium igazgatójának a Választmányhoz vagy a választmányi elnökhöz a cél megjelölésével írásban benyújtott indítványára legkésőbb tíz napon belül össze kell hívni. Amennyiben az összehívás tíz napon belül nem történik meg, akkor a kezdeményezők önállóan is összehívhatják a Közgyűlést.
	\item A Közgyűlés ülései az összehívás szempontjából minősülhetnek rendesnek és rendkívülinek.
	\begin{enumerate}
		\item A rendes Közgyűlés helyét, idejét és napirendjét a rendes Közgyűlést megelőzően legalább hét nappal közölni kell a tagokkal és a meghívottakkal.
		\item Sürgős ügyek megtárgyalására rendkívüli Közgyűlés hívható össze. A rendkívüli ülés helyét, idejét és napirendjét a rendkívüli Közgyűlést megelőzően legalább két nappal közölni kell a tagokkal és a meghívottakkal. A rendkívüli Közgyűlés nem dönthet a 4. § (2) a., b., c., d. pontokban felsorolt kérdésekről.
	\end{enumerate}
	\item A Közgyűlés akkor határozatképes, ha a tagok több mint fele jelen van.
	\begin{enumerate}
		\item A határozathozatalhoz a jelenlevő tagok egyszerű többségének egybehangzó szavazata szükséges, kivéve a 4. § (2) b. pontja, amelyben a jelenlevők kétharmadának egybehangzó szavazata szükséges.
		\item Titkos szavazásra kerül sor valamennyi személyi kérdésben, illetve ha azt a Közgyűlés legalább öt tagja kéri.
		\item Amennyiben az összehívott Közgyűlés nem határozatképes, akkor az ülést nyolc napon belül változatlan napirenddel össze kell hívni.
		\item A megismételt közgyűlés határozatképes, ha a tagok több mint negyede jelen van.
	\end{enumerate}
	\item Titkos szavazás esetén a Közgyűlés jelenlévő tagjai közül háromtagú Szavazatszedő és -számláló Bizottságot (továbbiakban: SZB) kell választani, melynek nem lehet tagja a Választmány tagja, műhelytitkár és kuratóriumi diáktag. A szavazás eredményéről a SZB írásos jelentést készít és ezt három munkanapon belül nyilvánosságra hozza. Ezt a jelentést csatolni kell a Közgyűlésről vezetett jegyzőkönyvhöz.
	\item A Közgyűlés valamennyi tagját indítványozási, véleménynyilvánítási, javaslattételi és szavazati jog illeti meg. A Közgyűlés minden tagja egy, át nem ruházható szavazattal rendelkezik.
	\item A Közgyűlés üléseiről jegyzőkönyvet kell készíteni, amely tartalmazza az ülés lefolyásának lényegesebb mozzanatait, továbbá a napirendi pontokkal és az indítványokkal kapcsolatos határozatokat, különös tekintettel a szavazásra feltett kérdésekben elhangzott kisebbségi véleményekre és szavazati arányokra.
	\begin{enumerate}
		\item A jegyzőkönyv elkészítéséről a Titkár gondoskodik, és azt az ülést levezető elnökkel, a CHÖK egy, az elnök által felkért tagjával együtt aláírásával hitelesíti (kivéve a b. pontban meghatározott esetben).
		\item A Titkár akadályoztatása esetén a levezető elnök a Közgyűlés tagjai közül jegyzőkönyvvezetőt kér fel.
		\item A jegyzőkönyvet 15 napon belül nyilvánosságra kell hozni.
		\item A jegyzőkönyv mellé minden esetben csatolni kell a jelenléti ívet is.
	\end{enumerate}
	\item A Közgyűlés ülései a Collegium valamennyi tagja számára nyilvánosak, de a levezető elnök valamely tag kérésére, egy napirendi pont megvitatására – a Közgyűlés által egyszerű többséggel hozott döntéssel – zárt ülést is elrendelhet.
	\begin{enumerate}
		\item Zárt ülés esetén mindenkinek, aki nem tagja a CHÖK-nek, el kell hagynia a termet.
		\item Zárt ülés csak és kizárólag azt a napirendi pontot tárgyalhatja, amely miatt elrendelték.
		\item A napirendi pont lezárása után a zárt ülést a levezető elnöknek föl kell függesztenie, és a Közgyűlés ismét nyílt üléssé alakul.
	\end{enumerate}
	\item Bármely tag javaslatára a Közgyűlés egyszerű szótöbbséggel tanácskozási jogot szavazhat meg a jelenlévő személyeknek.
	\item A Közgyűlés véleményezi az igazgatói pályázatokat (vö. SzMSz 40. §).
\end{enumerate}


\section{A Választmány}

\begin{enumerate}
	\item A CHÖK operatív vezető testülete és végrehajtó szerve a Választmány.
	\item A Választmány a 2. §-ban meghatározott célokat megvalósítandó a következő feladatokat végzi, illetve hatásköröket gyakorolja:
	\begin{enumerate}
		\item irányítja a CHÖK tevékenységét a Közgyűlés ülései közötti időszakokban,
		\item dönt minden olyan kérdésben, amelyet a Közgyűlés, egyetemi vagy önkormányzati szabályzat rá átruházott vagy részére megállapított, illetve amely nem tartozik a Közgyűlés kizárólagos hatáskörébe,
		\item gondoskodik a jelen Szabályzatban foglaltak, az SzMSz, a Tanulmányi Szabályzat, a CHÖK működési szabályzata, a Házirend, valamint a Közgyűlés határozatainak végrehajtásáról és betartásáról,
		\item folyamatos és szervezett kapcsolatot tart fenn más hallgatói szervezetekkel, kiemelten fontos ezek közül az ELTE EHÖK, BTK HÖK, IK HÖK, a TÁTK HÖK, TTK HÖK és az ELTE KolHÖK, a Szakkollégiumi Mozgalom, valamint ezek bizottságai, illetve a szakkollégiumok és testvérkollégiumok.
		\item kezdeményezi, megtervezi, irányítja és ellenőrzi az egyes Bizottságok működését.
	\end{enumerate}
	\item A Választmány munkájáról a Választmány elnöke és a Gazdasági Bizottság elnöke minden nem rendkívüli Közgyűlésen beszámol.
\end{enumerate}


\section{A Választmány tagjai}

\begin{enumerate}
	\item A Választmány kilenc rendes tagból áll:
	\begin{enumerate}
		\item a Választmány elnöke,
		\item a Választmány kettő alelnöke,
		\item a Gazdasági Bizottság elnöke,
		\item a Kommunikációs Bizottság elnöke,
		\item a Közösségi Bizottság elnöke,
		\item a Kulturális Bizottság elnöke,
		\item a Sportbizottság elnöke,
		\item a Tudományos Bizottság elnöke.
	\end{enumerate}
	\item A Választmány kötelezően meghívott tagja a Titkár és a Tanácsadó Testület tagjai és a kuratóriumi diáktagok.
	\item A Választmány mandátuma:
	\begin{enumerate}
		\item A Választmány mandátuma az egy évig tartó megbízatása lejártáig,  illetve a Választmány elnökének vagy alelnökeinek lemondásáig vagy a Közgyűlés által történő felmentéséig tart. Egy éves megbízatás esetén a Választmány mandátuma
		\begin{enumerate}
			\item tavaszi félévben történő megválasztás esetén a rákövetkező évbeli március 1-ig érvényés;
			\item őszi félévben történő megválasztás esetén a rákövetkező évbeli október 1-ig érvényés
		\end{enumerate}
		érvényes.
		\item A Választmány elnökének és alelnökeinek mandátuma megszűnik, amennyiben
		\begin{enumerate}
			\item az egy éves megbízatása lejár,
			\item lemond vagy a Közgyűlés felmenti.
		\end{enumerate}
		\item A bizottsági elnökök mandátuma megszűnik, amennyiben
		\begin{enumerate}
			\item az egyéves megbízatásuk lejár,
			\item lemond vagy a választmányi elnök a Választmány egyetértésével felmenti,
			\item a választmányi elnök vagy alelnökök mandátumának megszűnésével
		\end{enumerate}
		\item Amennyiben a Választmány megbízatása véget ér két Közgyűlés között, két héten belül rendes Közgyűlést kell összehívni, amely időpontjáig az előző Választmány ügyvezető Választmányként működik.
	\end{enumerate}
	\item A Választmány elnöke
	\begin{enumerate}	
		\item a Közgyűlés és a Választmány döntéseinek megfelelően, a hatályos jogszabályok, egyetemi szabályzatok és a Collegium szabályzatainak rendelkezései alapján irányítja, vezeti és egy személyben képviseli a CHÖK-öt,
		\item irányítja a Választmány tevékenységét,
		\item folyamatos munkakapcsolatot tart fenn a Collegium igazgatójával és más tisztségviselőivel,
		\item felel a Választmány és a Közgyűlés üléseinek előkészítéséért, 
		\item vezeti a Közgyűlés és a Választmány üléseit,
		\item tisztsége alapján képviseli a CHÖK-öt a 7. § 2. d.-ben meghatározott fórumokon,
		\begin{enumerate}
			\item Akadályoztatása esetén hivatalos meghatalmazással delegálhatja a Választmány tagjait. Kívánatos, hogy az alelnököket, illetve az illetékes bizottságok elnökeit delegálja egyes szervezetek üléseire.
		\end{enumerate}
		\item feladat- és hatáskörét átruházhatja a Választmány tagjaira egyedi megbízással, a megbízott az elvégzett munkáról a Választmánynak beszámolni köteles,
		\item felügyeli a Tudományos Bizottság működését,
		\item joga van rendkívüli feladatok ellátása miatt eseti (ad hoc) bizottság létrehozására,
		\item a Választmány delegáltja a KolHÖK üléseire.
	\end{enumerate}
	\item A Választmány alelnökei
		\begin{enumerate}
		\item a választmányi elnök távollétében a választmányi elnök feladatait valamely alelnök látja el.
		\item Az alelnökök felügyelik és koordinálják a Gazdasági, Kommunikációs, Közösségi, Kulturális és Sportbizottságok működését.
		\item Az alelnökök figyelemmel kísérik a Választmány és a 7. § 2. b.-ben meghatározott fórumok munkáját, kapcsolatot tart fent a Collegium és jelen szervek között, illetve szükség esetén elsősorban ők helyettesítik az elnököt ezen fórumokon.
		\item Az alelnökök az elnök akadályoztatása esetén a Választmány delegáltja a KolHÖK küldöttgyűlésébe.
		\item A mindenkori alelnökök az aktuális feladatokat egymás közötti kölcsönös megegyezés alapján osztják fel.
		\end{enumerate}
	\item A bizottsági elnök
		\begin{enumerate}
		\item a CHÖK szakmai és közösségi feladatait látja el a választmányi elnök irányítása és felügyelete mellett, utasításai alapján,
		\item a választmányi elnök nevezi ki a Collegium hallgatói közül,
		\item feladata:
			\begin{enumerate}
			\item segíti a választmányi elnök munkáját az illetékes szakterületen;
			\item irányítja az adott szakterület, illetve szaktestület munkáját;
			\item tevékenységéről havi rendszerességgel szóban köteles beszámolni a választmányi elnöknek a választmányi ülésen;
			\item tevékenységéről félévente írásban beszámol a Közgyűlésnek;
			\item szemeszterenként legalább kétszer összehívja és vezeti az adott szaktestület ülését, felelős a szaktestületi döntések végrehajtásáért;
			\item képviseli az érintett szaktestületet, a szaktestület munkájáról beszámol a Választmánynak és a Közgyűlésnek;
			\item elkészíti a feladatkörébe tartozó ügyek szabályzattervezetét, módosítását;
			\item folyamatosan kapcsolatot tart a szakterületén érdekelt más kari és egyetemi hallgatói önkormányzati tisztségviselőkkel, valamint collegiumi és egyetemi dolgozókkal;
			\item ellátja mindazokat a feladatokat, amit jelen Alapszabály vagy a választmányi elnök hatáskörébe utal.
			\end{enumerate}
		\end{enumerate}
	\item A bizottságok működése
		\begin{enumerate}
		\item A bizottságok a 8. § (8) – (14) pontban meghatározott feladatukat ellátják.
		\item A bizottsági üléseken mandátummal rendelkezik a bizottság elnöke és a bizottság tagjai, akiket a bizottság elnöke vagy valamely alelnök vagy az elnök javaslatára a Választmány elnöke nevez ki. 
		\item Minden bizottság ülésére kötelezően meghívott a Választmány elnöke és az alelnökök, akik tanácskozási joggal lehetnek jelen.
		\item A bizottsági üléseken tanácskozási joggal jelen lehetnek a CHÖK tagjai.
		\item A bizottság üléseiről jegyzőkönyv készül, amelyet a jegyzőkönyvvezető, a bizottság elnöke és egy bizottsági tag aláírásával hitelesít, majd a Titkár iktat.
		\item A bizottsági tagok száma a bizottsági elnökön kívül legalább két fő. 
			\begin{enumerate}
			\item Amennyiben létszámuk kettő alá csökken, a megüresedett helyekre a bizottság elnöke, a választmányi elnök vagy valamely alelnök által javasolt személyeket a Választmány elnöke nevezi ki.
			\end{enumerate}
		\end{enumerate}
	\item A Gazdasági Bizottság
	\begin{enumerate}
		\item felelős a CHÖK szabályszerű pénz- és vagyonkezeléséért, valamint a Gazdálkodási és Vagyongazdálkodási szabályzat (ELTE Szervezeti és Működési Szabályzat I. 5. melléklet) betartásáért,
		\item kezeli a Választmány vagyonát,
		\item félévente beszámolót készít, amelyet a Közgyűlésen ismertet a CHÖK tagjaival,
		\item felel a nagyobb pályázatok megírásáért és elszámolásáért,
		\item felügyeli a Választmány és a bizottságok által írt pályázatok elkészítését, elszámolását,
		\item szükség esetén segítséget nyújt az elkészítésben, elszámolásban,
		\item elkészíti a Választmány szemeszterenkénti költségvetési tervét,
		\item segít a többi Bizottságnak a programok költségvetési tervét elkészíteni, illetve felügyeli, ellenőrzi és jóváhagyja ezeket,
		\item új pénzforrásokat, pályázati forrásokat és szponzorációs lehetőségeket keres,
		\item iktatja a pénzügyeket, ezáltal transzparenssé téve a Választmány pénzgazdálkodását,
		\item felel a pályázatfigyelésért, illetve a Bizottságok ilyen irányú tájékoztatásáért,
		\item kapcsolatot tart az ELTE Karrierközponttal, a Pályázati és Innovációs Központtal, lehetőség szerint igényeknek megfelelő képzéseket szervez a Collegiumba,
		\item kapcsolatot tart az Igazgatóval a Collegium gazdálkodásával kapcsolatos kérdésekben, valamint a nagyobb pályázatok ügyében.
		\end{enumerate}
	\item A Kommunikációs Bizottság
	\begin{enumerate}
		\item rendszeresen tájékoztatja a Collegiumot, a CHÖK tagjait a Választmány munkájáról,
		\item felelős a Collegium külső megjelenéséért, reprezentációjáért, a Collegium programjainak hirdetéséért,
		\item felel a Választmány kiadványainak korrektúrázásáért,
		\item moderálja a Választmány hatáskörébe tartozó levelezőlistákat és elektronikus felületeket,
		\item további megjelenési és hirdetési felületeket keres,
		\item a Választmánnyal egyeztetve kapcsolatot tart az ELTE BTK, IK, TÁTK és TTK kommunikációs bizottságaival, valamint más ELTE-s kommunikációs fórumokkal,
		\item profiljába tartozó eszközökre, szolgáltatásokra pályázik.
	\end{enumerate}
	\item A Közösségi Bizottság
	\begin{enumerate}
		\item felel a közösségi, gólya- és alumniprogramok, projektek megszervezéséért és lebonyolításáért,
		\item a Collegium szakmai programjaira biztosítja a humán erőforrásokat és a szükséges feltételeket (étel, ital, promóciós termékek, kiadványok, ajándékok stb.),
		\item felel a közösségi élet dokumentációjáért (fényképek, videók, plakátok stb.),
		\item együttműködik az önszerveződő közösségi körökkel, csoportosulásaival és segíti munkájukat,
		\item profiljába tartozó rendezvényekre, eszközökre, szolgáltatásokra pályázik.
	\end{enumerate}
	\item A Kulturális Bizottság
	\begin{enumerate}
		\item segíti a Collegium kulturális körei, csoportosulásai munkáját,
		\item igényfelmérés alapján kulturális programokat szervez,
		\item profiljába tartozó rendezvényekre, eszközökre, szolgáltatásokra pályázik.
	\end{enumerate}
	\item A Sportbizottság
	\begin{enumerate}
		\item segíti a Collegiumban működő sportkörök munkáját, sportprogramokat szervez,
		\item elősegíti az esélyegyenlőség megvalósulását,
		\item tartja a kapcsolatot a profiljába tartozó szervezetekkel, egyesületekkel,
		\item a profiljába tartozó rendezvényekre, eszközökre, szolgáltatásokra pályázik.
	\end{enumerate}
	\item A Tudományos Bizottság
	\begin{enumerate}
		\item két szervből áll: a háromfős operatív szervből és a kiterjesztett bizottságból,
		\item az operatív szerv tagjai: a bizottság elnöke és a két bizottsági tag,
		\item az operatív szerv felel a kiterjesztett bizottság üléseinek előkészítéséért, a döntések, határozatoknak megfelelően a kivitelezésért, megszervezéséért; kapcsolatot tart az EDÖK-kel és az ELTE EHÖK, BTK, IK, TÁTK, TTK Tudományos Bizottságaival.
		\item a kiterjesztett bizottság mandátummal rendelkező rendes tagjai a Collegium műhelyeinek választott titkárai, és a bizottság elnöke. A műhelytitkárok akadályoztatásuk esetén a kiterjesztett bizottsági ülésekre írásos meghatalmazással műhelyükből egy főt delegálhatnak.
		\item az operatív szerv ülését és a kiterjesztett bizottsági ülést kéthavonta felváltva hívja össze a bizottság elnöke. A kiterjesztett bizottsági ülésen a bizottság elnöke és az operatív szerv egy tagja kötelezően részt vesz. A Választmány elnöke kötelezően meghívott.
		\item felelős a collegiumi szintű szakmai programok megszervezéséért, az azokkal kapcsolatos elvi döntések meghozataláért,
		\item az egyes műhelyek szabályzatmódosításában véleményezési joga van,
		\item a műhelytitkárok igényei szerint felel a műhelyeket érintő problémák megoldásáért, a műhelyprogramok hirdetéséért.
	\end{enumerate}
	\item A Titkár
	\begin{enumerate}
		\item felügyeli a Választmány szabályszerű működését, ellenőrzi tagjainak tevékenységét,
		\begin{enumerate}
			\item Amennyiben a Titkár szabálytalanságot észlelt, azt köteles az elnöknek jelenteni.
			\item Amennyiben a szabálytalanság megismétlődik, illetve a Választmány nem mutat kellő igyekezetet annak javítására, a Titkár köteles a szabályszegést a Közgyűlésnek jelenteni.
		\end{enumerate}
		\item meghívottként részt vesz a Közgyűlésen, a Választmány és az egyes bizottságok ülésein, azokról jegyzőkönyvet készít,
		\item akadályoztatása esetén a választmányi elnök vagy az adott bizottság elnöke jegyzőkönyvvezetőt kér fel,
		\item a Titkár vagy a helyette kinevezett jegyzőkönyvvezető a Közgyűlésen, a Választmány és a bizottságok ülésein készült jegyzőkönyveket aláírásával hitelesíti és iktatja.
	\end{enumerate}
	\item A Tanácsadó Testület
	\begin{enumerate}
		\item tagja tapasztalt collegisták, illetve volt collegisták, akik collegiumi éveik során felgyülemlett tudással, tanácsaikkal segítik a Választmány munkáját.
		\item tagja lehet minden olyan (volt) collegista, akinek legfeljebb öt éve szűnt meg a collegiumi tagsága, kivéve, ha az kizárás útján történt meg,
		\item tagjait (minimum 1 fő) a Választmány egyetértésével a Választmány elnöke kéri fel,
		\item tagjai kötelezően meghívottként, tanácskotási joggal részt vesznek a Választmány és a Közgyűlés ülésein,
		\item tisztségük az elnök mandátumának lejártáig érvényes.
	\end{enumerate}
\end{enumerate}


\section{A Választmány működése}

\begin{enumerate}
	\item A Választmány üléseit az elnök, akadályoztatása esetén valamelyik alelnöke hívja össze, készíti elő és vezeti le.
	\item A Választmány üléseit szorgalmi időszakban legalább havonta egyszer össze kell hívni. Rendkívüli esetben, ha azt az elnök vagy a Választmány legalább két tagja szükségesnek tartja, az igény írásbeli, a választmányi elnök hivatalos e-mail címére küldött bejelentésétől számított 3 napon belül össze kell hívni.
	\item A Választmány ülésein kerül sor a Választmány hatáskörébe tartozó feladatok kiosztására, a bizottságok munkájának összehangolására, valamint a Választmány tagjai által benyújtott felterjesztések megvitatására és megszavazására.
	\item A választmányi elnök, az alelnökök és a bizottsági elnökök a Választmány ülésein kötelesek beszámolni az azt megelőző választmányi ülés óta eltelt időben végzett munkájukról.
	\item A bizottsági elnökök a kinevezett bizottsági tagok közül hivatalos meghatalmazással delegálhatnak egy főt a Választmány ülésére.
	\item A bizottsági tagokat a választmányi elnök, valamely alelnök vagy a bizottsági elnök javaslatára a Választmány egyszerű szótöbbséggel választja meg.
	\item A Tanácsadó Testület tagjait a választmányi elnök vagy valamely alelnök javaslatára a Választmány egyszerű szótöbbséggel való javaslata után a választmányi elnök kéri fel.
	\item A Választmány ülései nyilvánosak, annak helyét, idejét és napirendi pontjait a választmányi ülést megelőzően 2 nappal előre közölni kell a Választmány és a CHÖK tagjaival és a meghívottakkal.
	\item A Választmány ülesein a CHÖK minden tagja tanácskozási joggal jelen lehet.
	\item A Választmány a választmányi ülésen egyszerű szótöbbséggel a választmányi elnök vagy alelnökök javaslatára felmentheti tisztségéből a bizottságok elnökeit. Ebben az esetben a megüresedett posztot a választmányi elnök javaslatára 10 munkanapon belül a Választmány határozata alapján új bizottsági elnök tölti be. Az új bizottsági elnök hivatalba lépéséig feladatait a választmányi elnök vagy az illetékes alelnök látja el.
	\item Határozathozatal a Választmányban:
	\begin{enumerate}
		\item A választmányi ülés határozatképes, ha tagjainak több mint fele jelen van. A határozathozatalhoz a jelenlevők egyszerű többségének egybehangzó szavazata szükséges. Szavazategyenlőség esetén az elnök szavazata dönt.
		\item Határozatképtelenség esetén legkésőbb 8 napon belül megismételt választmányi ülést kell összehívni, amely minden esetben határozatképes.
		\item Titkos szavazásra kerül sor:
		\begin{enumerate}
			\item valamennyi személyi kérdésben,
			\item ha azt a Választmány legalább két tagja kéri.
		\end{enumerate}
		\item A választmányi elnök zárt ülést rendelhet el egy konkrét napirendi pont megvitatására. Ez esetben minden nem választmányi tagnak el kell hagynia a termet. A napirendi pont megvitatása után a választmányi ülés ismét nyilvános.
	\end{enumerate}
	\item A Választmány üléseiről jegyzőkönyvet kell készíteni, melyet a jegyzőkönyvvezető, az elnök és a Választmány egyik tagja aláírásával hitelesít, és a Titkár iktat.
\end{enumerate}


\section{A Választmány választása}

\begin{enumerate}
	\item A Választmány elnökét, alelnökeit és a bizottságok elnökeit a Közgyűlés választja listás szavazással.
	\item A bizottsági tagokat a bizottsági elnök javaslatára az alelnökök egyetértésével a Választmány elnöke nevezi ki a Választmány alakuló ülésén. A kinevezésekről jegyzőkönyv készül.
	\item A választás lebonyolítására a választó Közgyűlésen, a leköszönő Választmány beszámolóinak meghallgatása után kerül sor. A választó Közgyűlést minden évben legkésőbb október 1-jéig össze kell hívni.
	\item A jelöltállítás:
	\begin{enumerate}
		\item A CHÖK minden tagja választhat és választható. Listát bárki állíthat, a listán a Választmány elnöke, alelnökei és a bizottságok elnökei kötelezően szerepelnek. A listát az elnökjelöltnek a tanévnyitó Közgyűlés kihirdetését követő öt napon belül a Közgyűlést kihirdető választmányi elnök hivatalosan iktatott címére el kell juttatnia, a választmányi elnök pedig azt a kézhezvételétől számított 24 órán belül köteles a CHÖK minden tagja számára nyilvánosságra hozni.
		\item A lista állításával a Közgyűlésen az elnökjelölt, mint a listán szereplő személyek képviselője, nyilatkozni köteles, hogy a listán szereplő személyeknek a kívánt tisztség betöltését akadályozó körülményről nincs tudomása, és hogy azok vállalják a jelölést.
		\item A listán szereplő személyekről a Közgyűlés egyetlen szavazással, egyszerű többséggel dönt. A szavazólapon a listák valamennyi tagjának neve szerepel a 8. § 1.-ben meghatározott sorrendben. A Közgyűlés minden mandátummal rendelkező tagja legfeljebb egy listára szavazhat.
	\end{enumerate}
	\item A választás eredményét 7 munkanapon belül meg kell küldeni a Collegium igazgatójának, a rektornak, valamint nyilvánossá kell tenni a CHÖK és a Collegium más intézményei tagjainak, az EHÖK-nek és a KolHÖK-nek. (vö. ESZ. 22. §)
\end{enumerate}


\section{A Kuratórium diáktagjainak választása}

\begin{enumerate}
	\item A Collegium Kuratóriuma diáktagjainak számáról (3 fő) az SZMSZ rendelkezik. Választásuk jelen szabályzat értelmében a következőképp történik: 
	\begin{enumerate}
		\item A tagokat a Közgyűlés választja.
		\item A CHÖK minden tagja választ és választható. Jelöltet bármely CHÖK tag állíthat.
		\item A szavazás titkos. Érvényességéhez legalább 4 jelölt állítása szükséges. Ennek hiányában a Közgyűlést egy hónapon belül meg kell ismételni.
		\item Egy személy legfeljebb 3 jelöltre szavazhat.
		\item A három legtöbb szavazatot kapott jelölt a Kuratórium rendes tagja, a negyedik a Kuratórium póttagja lesz.
		\item A diáktag megbízása egy évig (októbertől októberig), lemondásig vagy visszahívásig tart.
		\item A póttag valamely rendes tag távolmaradása esetén köteles megjelenni a Kuratórium ülésén. Ebben az esetben szavazati joga is van. A póttag valamely rendes tag lemondása, visszahívása esetén automatikusan annak helyébe lép. Ilyenkor új póttagot kell választani, akinek mandátuma a következő februárig, lemondásig vagy visszahívásig tart.
		\item A kuratóriumi diáktagok a Kuratórium ülése utáni első választmányi ülésen kötelesek beszámolni a Kuratóriumban végzett munkájukról.
	\end{enumerate}
\end{enumerate}


\section{A műhelytitkár}

\begin{enumerate}
	\item A műhelyek működését az SzMSz 40–48. § szabályozza.
	\item A műhelytitkár
	\begin{enumerate}
		\item a műhelytagok által a félévnyitó műhelygyűlésen egy évre megválasztott tisztségviselő (vö. SzMSz 47. § (1) e.),
		\item feladata a műhely tagságának képviselete és tájékoztatása,
		\item tisztsége alapján mandátummal rendelkezik a Tudományos Bizottság kiterjesztett ülésein,
		\item a műhelygyűlésen tisztségéből visszahívható,
		\item tisztsége megszűnik collegiumi tagsága megszűntével, visszahívása vagy lemondása esetén.
	\end{enumerate}
\end{enumerate}


\section{A CHÖK irodája}

\begin{enumerate}
	\item A CHÖK feladatainak ellátására a CHÖK irodát működtet.
	\item Az iroda szervezetét és működési rendjét a bizottsági elnökök és az elnök javaslatára a választmányi ülés fogadja el.
	\item Az iroda szervezeti és működési rendjéért a Választmány a felelős.
	\item Az iroda:
	\begin{enumerate}
		\item a közösségi élet elengedhetetlenül fontos kellékeit hivatott tárolni, melyeket más helységben elhelyezni nem lehet;
		\item információs hátteret biztosítanak a hallgatók számára, vagyis a korábbi jegyzőkönyveknek, az aktuális szabályzatoknak benn olvasható példányait tárolják.
	\end{enumerate}
	\item Az iroda házirendje a jelen Alapszabály mellékletének minősül.
\end{enumerate}


\section{Zárórendelkezések}

\begin{enumerate}
	\item Jelen szabályzat elfogadása után azonnal hatályba lép.
	\item Jelen szabályzatot a Közgyűlés 2018. szeptember 20-án elfogadta.
	\item Jelen szabályzat egy példányát a Collegium könyvtárában el kell helyezni, valamint a Collegium honlapján elérhetővé kell tenni.
\end{enumerate}
\end{document}
