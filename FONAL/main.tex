\documentclass{rulebook}

\begin{document}
\section*{ELTE Eötvös József Collegium \\ ,,ALFONSÓ'' nyelvi programja \\ \vspace{0.5em} A collegiumi nyelvtanulás szabályozásáról} 

\vspace{2em}


\section*{\normalfont Preambulum} 

Az Eötvös József Collegium hivatása, hogy a legtehetségesebb magyar egyetemi polgárokat támogassa, tudományos előmenetelüket segítse. Célja, hogy olyan kiválóan felkészült szakembereket képezzen, akik tudományterületükön átlagot meghaladó tudással rendelkeznek, önálló kutatómunkára képesek, és akiknek a tudomány művelése nem csupán szakma, hanem tanári hivatás is.
A Collegium oktatási célkitűzései között – a hagyományoknak megfelelően -- egyszerre van jelen a széles körű, európai horizontú tudományos tájékozódás és a nemzeti hagyományok ápolásának igénye.
Fentiek szellemében a Collegium 2012 szeptemberétől az irányító és szabályozó testületek (az igazgató, a tanári kar és a hallgatói önkormányzat egésze, a közgyűlés valamint a kuratórium) egyetértő támogatásával unanimiter bevezette a nyelvtanulási lehetőségeket és kötelezettségeket leíró szabályrendszert. A szabályozást – ALFONSÓ keretprogram címen -- az első évek tapasztalatai és eredményei alapján jelen dokumentumban kiegészítettük, bővítettük és pontosítottuk.
Az Eötvös József Collegium az Eötvös Loránd Tudományegyetem tehetséggondozási munkájában betöltött történelmi szerepének megfelelően -- mindenkori lehetőségeihez mérten -- a nyelvtanulási keretprogramját a társ-szakkollégiumok tagsága felé is megnyitja.  
Óhajtjuk és bízunk abban, hogy a Collegiumtól – a magyar felsőoktatásban egyedülálló módon – kínált lehetőséggel a collegisták élni tudnak és tanulmányaikat európai szellemiséggel a nemzetközi kapcsolatrendszerünk kiszélesítésére, végső soron a magyar tudomány (oktatás és kutatás) erősítésére kamatoztatni fogják.  


\section{Alapelvek}

\begin{enumerate}
	\item	A collegiumi nyelvtanulás elsődleges célja, hogy a collegisták a program meghatározott időkereten belül legalább kettő, legalább B2-es nyelvvizsgával rendelkezzenek.
	\item Abban az esetben, ha a collegista nyelvvizsga nélkül érkezik, elégséges egy B2-es nyelvvizsgát tennie.
	\item	A collegistáktól elvárható a folyamatos fejlődésre való törekvés. Ennek megfelelően amennyiben az (1) pontban leírt feltétel már a felvétel idején teljesül, életbe lép a „szintlépési” kötelezettség: nyelvvizsga szerzése egy már ismert nyelvből C1-es szinten, vagy egy új nyelvből B2-es szinten, tekintet nélkül arra, hogy a hallgató hány nyelvvizsgával rendelkezik. 
	\item	Amennyiben az (1)--(3) pontokban meghatározott feltétel a rögzített határidő előtt teljesül, a Collegium erősen szorgalmazza, hogy collegista további szintlépést, vagy új nyelv elsajátítását tűzze ki célul. Ezen tanulmányok természetesen a mesterképzés idején is folytatódhatnak, és -- jóllehet már nem tartoznak a jelen szabályzatba foglaltak hatálya alá -- értékelési tényezőt jelentenek a mesterképzésbe való jelentkezések elbírálásában. %TODO utóbbi nem illik ide
\end{enumerate}

\section{Nyelvtanulási követelmények}
%TODO jelenleg csak az óralátogatási kötelezettség szűnik meg a szintfelmérők teljesítésével, a kimeneti követelmények nem szűnnek meg.

\begin{enumerate}
	\item A collegiumi nyelvtanulási modell keretében választható nyelvek: Angol, Latin, Francia, Olasz, Német, Spanyol, Ógörög – a továbbiakban: ALFONSÓ-nyelvek. 
	\item A collegiumi nyelvoktatásban való részvétel minden első-, másod- vagy harmadéves, újonnan felvett, BA/BSc-képzésben vagy osztatlan tanárképzésben résztvevő collegista számára kötelező.
	\item A collegisták számára új nyelvvizsga szerzésére három, szintemelésre kettő év áll rendelkezésre. 
	\begin{enumerate}
	\item Kivételt képeznek utóbbi alól azon osztatlan tanárképzésben részt vevő hallgatók, akiknek legalább az egyik tanári szakja nyelvszak, számukra három év áll rendelkezésre szintlépésre. %TODO miért?
	\end{enumerate}
	\item Általános követelmények tanulmányaikat nem nyelvi szakon folytató collegisták esetében, ha a collegistának
	\begin{enumerate}
		\item nincs nyelvvizsgája, a kimeneti követelmény egy legalább B2-es szintű nyelvvizsga;
		\item egy nyelvvizsgája van, a kimeneti követelmény egy újabb legalább B2-es szintű nyelvvizsga valamely más nyelvből.
		\item kettő vagy több nyelvvizsgája van, a kimeneti követelmény egy újabb legalább B2-es szintű nyelvvizsga valamely más nyelvből, vagy egy korábbi nyelv fejlesztésre C1-es szintre.
	\end{enumerate}
	\item A tanulmányaikat nyelvi szakon folytató collegisták részére az általános követelmény megegyezik a (4) pontban felsoroltakkal, azzal a kiegészítéssel, hogy 
	\begin{enumerate}
		\item a hallgató a fő- vagy minorszakjának megfelelő nyelvet a Collegiumban idegen nyelvként (vagyis az ALFONSÓ-program keretében) semmilyen formában nem tanulhatja;
		\item a hallgató a fő- vagy minorszakjának megfelelő nyelvből szerzett új nyelvvizsgát nem fogadtathatja el az ALFONSÓ-program követelményeként.
	\end{enumerate}
	\item Amennyiben egy adott nyelvből hozott vizsga nem tartozik a ALFONSÓ-nyelvek közé, a szintlépés lehetősége nem áll fenn. %TODO töröljük
	\item Az újonnan felvételt nyerő collegisták nyelvtudásának és nyelvtanulási terveinek regisztrációját, valamint a hallgatók „vállalásának” és nyelvi tanulmányaik folyamatosságának felügyeletét a műhelytitkárokkal való szoros együttműködésben a Választmány Tudományos Bizottsága koordinálja, és mindezekről a nyelvi munkaközösség-vezető tanárnak és az igazgatónak félévente áttekintő jelentést készít. %TODO ez nem így van
\end{enumerate}


\section{A számonkérés rendje}

\begin{enumerate}
	\item 	Új nyelvet kezdő collegisták esetében a hallgatók nyelvi fejlődésének ellenőrzésére a Collegium a tanévek végén a négy fő nyelvi kompetenciát vizsgáló szintfelmérést végez, ha
	\begin{enumerate}
		\item  az elérni kívánt szint B2, akkor az első év végén általánosságban A2-es szintű, a második év végén B1-es szintű, míg a harmadik év végén B2-es szintű feladatsorokkal.
		\item az elérni kívánt szint B2-ről C1, akkor az első év végén általánosságban B2-es és C1-es közötti, a második év végén C1-es szintű feladatsorokkal. 
	\end{enumerate}
	\item A tesztek nehézségi szintje az egyes nyelvek sajátosságait szem előtt tartva ettől adott esetben némileg eltérhet, de a vizsgáztatás során a vizsgáztatóknak alapvetően figyelembe kell venniük a vizsgát megelőző kurzusok tananyagát is.
	\item A felmérést írásbeli és szóbeli komplex vizsga alkotja.
	\item  Az írásbeli vizsga (teszt és/vagy hallás utáni szövegértés) a kurzusvezető szervezésében és irányításával zajlik.
	\item  A szóbeli vizsgabizottság legalább két főből áll, ezek
	\begin{enumerate}
		\item a vizsgára felkészítő kurzus oktatója;
		\item nyelvi műhely vezetője vagy titkára, felsőbbéves műhelytag, senior hallgató vagy diákképviseleti tag.
	\end{enumerate}
	\item A szintfelmérés keretét egy ,,nullkredites'' vizsgakurzus alkotja, melyet minden a programban részt vevő hallgató köteles felvenni.
	\item A szintfelmérő sikeresen teljesítettnek minősül, amennyiben a collegista eléri a 60\%-os ponthatárt. A collegista
	\begin{enumerate}
		\item 60–80\% közötti teljesítményéért „4 (jó)” értékelés jár;
		\item 80–100\% közt pedig „5 (jeles)” érdemjegy jár. diákképviseleti tag.
	\end{enumerate} %TODO és a többiért mi jár? 
	\item A szintfelmérést a collegiumi nyelvtanárok koordinálják.
\end{enumerate}


\section{Óralátogatási kötelezettség}

\begin{enumerate}
	\item A Collegium biztosította nyelvórákon a részvétel kötelező. 
	\item Az (1) kötelezettsége alól az igazgató eseti felmentést adhat.
	\item A collegiumi nyelvi kurzusok látogatása a hallgató felvételekor általa vállalt célkitűzések eléréséig, vagyis a nyelvvizsga/nyelvvizsgák megszerzéséig illetve bemutatásáig, vagy a collegiumi szabályozás tekintetében ezekkel egyenértékűnek tekintett szintfelmérő vizsgák teljesítéséig kötelező marad.
	\item A rendszeres óralátogatást az adott kurzusvezető ellenőrzi és dokumentálja.
	\item A hallgató a tanórák legalább 75\%-án köteles megjelenni.
\end{enumerate}


\section{A latin nyelv tanulásáról}

\begin{enumerate}
	\item Első collegiumi évében kötelező a latin nyelv tanulása minden
	\begin{enumerate}
		\item alapképzésben bölcsész- és társadalomtudományi karra járó collegistának;
		\item osztatlan tanárképzésben bölcsész alapkaros első három évében felvett collegistának.
	\end{enumerate}
	\item A Filozófia Műhely hallgatói választhatnak a latin és az ógörög nyelv között.
\end{enumerate}


\section{Egyéni elbírálás}

\begin{enumerate}
	\item Minden olyan egyedi esetben, amelyről e szabályzat nem rendelkezik, a hallgató nyelvtanulási kötelezettségének teljesítésére vonatkozó kérését egyéni elbírálás alá eső kérvény formájában nyújthatja be a Collegium igazgatójához.
	\item Egyéni esetnek minősül a
	\begin{enumerate}
		\item  szabályzatban rögzítettektől való szintbeli eltérés;
		\item az ALFONSÓ-nyelvek körébe nem tartozó nyelvekből szerzett nyelvvizsgák elfogadtatása. 
	\end{enumerate}
	\item Az ALFONSÓ program keretében sikeresen teljesített nyelvi szintfelmérő vizsgák az egyéni kérvények elbírálása során figyelembe veendők.
\end{enumerate}

\section{Zárórendelkezések}

\begin{enumerate}
	\item Jelen szabályzat az ELTE Eötvös József Collegium Oktatási, Tanulmányi Szabályzat és Követelményrendszer mellékletét képezi.
	\item Jelen szabályzat 2018. február 22-től érvényes.
\end{enumerate}

\end{document}