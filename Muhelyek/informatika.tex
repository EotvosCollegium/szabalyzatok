\documentclass{../styles/rulebook}

\begin{document}
\section*{ELTE Eötvös József Collegium \\ Informatika műhely\\ \vspace{0.5em} Műhelyszabályzat} 

\vspace{2em}

\section*{Preambulum}

Az ELTE Eötvös József Collegium (a továbbiakban Collegium) nagymúltú hagyományainak, célkitűzéseinek meg-felelően, a 2004-ben alapított Informatikai Műhely (a továbbiakban Műhely) a tudós, önálló kutatásaikkal is kitűnő, hivatásukat szerető és szívvel-lélekkel végző informatikusok képzését tekinti feladatának. Az Informatikai Műhely annak a felismerésnek a szellemében fogant, miszerint a XXI. századi informatikus képzésben feltétlenül szükséges a legújabb elméleti és gyakorlati eredmények megismerése és a modern informatika egységes, átfogó ismerete.

\section{Célkitűzések}

\begin{enumerate}
	\item A Műhely munkájának célja az informatika és az informatikával kapcsolatos szakú collegista hallgatók képzésének kiegészítése. Ebből egyértelműen következik, hogy a collegistáknak többet kell teljesíteniük nem collegista évfolyamtársaikhoz képest. E többletteljesítmény kereteit a Műhely jelen szabályzata határozza meg.
	\item A collegiumi előadások, gyakorlatok és szemináriumok többségét az ELTE Informatikai Karának tanszékei saját tanegységükként ismerik el.
	\item A Collegium és az École Normale Supérieure történelmi kapcsolatai és az ezekben rejlő tanulmányi lehetőségek miatt az Informatikai Műhely tagjainak figyelmébe ajánljuk a francia és a német nyelv tanulását is.
\end{enumerate}


\section{Szabályok}

\begin{enumerate}
	\item A Műhely az informatikai szakterület legtekintélyesebb magyar tudósai közül tiszteletbeli elnököt választhat.
	\item A Műhely tagja a Collegiumban főállásban vagy óraadóként informatikai tanegységet oktató tanár, valamint minden olyan bentlakó vagy bejáró collegista, akinek legalább az egyik szakja az Informatikai Karhoz tartozik. 
	\item A Műhely minden tanévben két, szemeszterenként egy-egy műhelygyűlést tart, amelyek egyikét lehetőség szerint kirándulással köti egybe.
	\item A Műhely collegista tagjai minden évben a tanév első műhelygyűlésén maguk közül szótöbbséggel titkárt választanak.
	\item A Műhely szemeszterenként megújított előadás, speciális kollégium, illetve szeminárium kínálatát a Műhely vezetője a Műhely tagjainak véleményét meghallgatva alakítja ki.
	\item A collegisták a szemeszterenként meghirdetett előadások, speciális kollégiumok, szemináriumok közül legalább egy, szakjuknak vagy szakjaiknak megfelelő, az ELTE által is meghirdetett tárgyat kötelesek leckekönyvükbe is felvenni és abból levizsgázni. 
	\item A Műhely által meghirdetett előadásokat, speciális kollégiumokat, szemináriumokat minden collegista felveheti. 
	\item Az Informatikai Műhely tagjai más Műhely által meghirdetett előadásokat, speciális kollégiumokat, szemináriumokat is felvehetnek, és ezek sikeres elvégzése esetén, a Műhely vezetőjének beleegyezésével a (6) pontban megfogalmazott követelményt kiválthatják.
	\item Az egyes alkalmakon a tanár engedélyével külsős hallgatók is részt vehetnek. A collegiumi órák lehetőség szerint a Collegium épületében legyenek.
	\item A Collegium biztosította tutoriális képzést kihasználva az Informatikai Műhely tagjai egyetemi tanulmányaik alatt, de legkésőbb a nyolcadik félév végéig, legalább egy olyan dolgozatot kötelesek írni, melyet az Informatikai Kar Tudományos Diákköri Konferenciájára is benyújtanak.
	\item Amennyiben a fenti követelményeknek egy bentlakó vagy bejáró collegista nem tesz eleget, a Collegium tagjai közül a Collegium Tanári Karának és Igazgatójának egyetértésével kizárható.
\end{enumerate}


\section{Zárórendelkezések}

\begin{enumerate}
	\item Jelen műhelyszabályzat 2004. február 15-től lép életbe.
	\item Jelen szabályzat a Collegium Oktatási, Tanulmányi Szabályzatának és Követelményrendszerének mellékletét képezi.
\end{enumerate}

\end{document}