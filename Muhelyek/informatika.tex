\documentclass{../styles/rulebook}

\usepackage{lmodern}

\frenchspacing
\titlespacing{\section}{0pt}{8ex}{5ex}
\setlist[enumerate,1]{label={(\arabic*}), itemindent=\labelwidth,  leftmargin=0pt , ref=\arabic*, labelsep=9pt}

\begin{document}
\section*{ELTE Eötvös József Collegium \\ Informatikai Műhely\\ \vspace{0.5em} Műhelyszabályzat}

\vspace{2em}

\section*{Preambulum}

Az Informatikai Műhely (a továbbiakban: Műhely) az Eötvös Loránd Tudományegyetem (a továbbiakban: ELTE) Eötvös József Collegiumának (a továbbiakban: Collegium) szakmai és hallgatói közössége.
A Collegium nagy múltú hagyományainak, célkitűzéseinek megfelelően, a 2004-ben alapított Műhely a tudós, önálló kutatásaikkal is kitűnő, hivatásukat szerető és szívvel-lélekkel végző informatikusok képzését, tehetséggondozását tekinti feladatának. A Műhely annak a felismerésnek a szellemében fogant, miszerint a XXI. századi informatikus képzésben feltétlenül szükséges a legújabb elméleti és gyakorlati eredmények megismerése és a modern informatika egységes, átfogó ismerete.

\section{Célkitűzések}

\begin{enumerate}
	\item A Műhely elsődleges célja az, hogy segítse az ELTE Informatikai Kar legtehetségesebb hallgatóinak szakmai és tudományos előmenetelét, melynek legfőbb eszközei:
	      \begin{enumerate}
		      \item műhelykurzusok meghirdetésével a műhelytagok eredeti képzésének kiegészítése;
		      \item kutatószeminárium szervezése, ahol a Műhelytagok megismerhetik egymás kutatásait, és támogatást kaphatnak sajátjukhoz.
	      \end{enumerate}
	\item Továbbá összetartó, együttműködő közösség kialakítása.
\end{enumerate}

\section{A Műhely tagjai}

\begin{enumerate}
	\item A Műhelybe az ELTE Informatikai Karának hallgatói nyerhetnek felvételt.
	      \begin{enumerate}
		      \item A Műhely vezetőjének engedélyével az ELTE egyéb hallgatói is felvételt nyerhetnek.
	      \end{enumerate}
	\item A műhelytagok felvételének részleteiről a Collegium felsőbb szabályzatai rendelkeznek az alábbi kiegészítésekkel:
	      \begin{enumerate}
		      \item a szakmai bizottságnak a Műhely vezetője (vagy megbízottja) mellett tagja a műhelytitkár (akadályoztatása esetén más, a műhelyvezető által felkért műhelytag), a további tagokat a Műhely vezetője kéri fel;
		      \item a Collegium általános felvételi eljárását a Műhely egy írásbeli, szakmai feladatsorral egészítheti ki, melynek megírására a szakmai elbeszélgetések előtt kerül sor, és eredménye elsősorban a szakmai bizottság munkáját segíti.
	      \end{enumerate}
	\item A műhelytagság megszűnik:
	      \begin{enumerate}
		      \item a collegiumi státusz megszűnésével automatikusan;
		      \item a műhelytag a Műhely vezetőjének címzett írásos lemondásával;
		      \item a műhelyszabályzatban foglalt követelmények nem teljesítése esetén a Műhely vezetőjének döntése alapján.
	      \end{enumerate}
	\item A műhelytagoknak törekedniük kell a Collegium és a Műhely közösségi életében való aktív részvételre.
\end{enumerate}

\section{A Műhely tanulmányi rendje}


\begin{enumerate}
	\item Amennyiben a szabályzat a továbbiakban nem köt ki mást, úgy az egyetemi év a képzésen aktív évek száma, mesterszak esetén pedig az alapképzéssel együtt számolandó. A Collegiumba nem első egyetemi félévükben felvettekre a követelmények a felvétel félévétől várhatóak el.
	      \begin{enumerate}
		      \item A Műhely aktív felsőbb éves collegista tagjának számítanak a Műhely azon alapképzésen, mesterképzésen vagy osztatlan képzésen hallgató tagjai, akik a Collegiumban legalább harmadik aktív félévüket töltik; valamint a Műhely senior tagjai.
	      \end{enumerate}
	\item A Műhely tanulmányi rendjével törekszik a műhelytagok kiemelkedő szakmai és tudományos tevékenységét segíteni és ösztönözni.
	\item A műhelytagok tanulmányi és szakmai kötelességei:
	      \begin{enumerate}
		      \item Minden félévben a műhely által meghatározott tárgykínálatnak megfelelően kell tárgyat vagy tárgyakat elvégezniük.
		      \item Pályázniuk kell a kari \emph{Neumann János Tehetséggondozó Kör}be, amennyiben jogosultak rá.
		      \item A műhelytagoknak
		            \begin{enumerate}
			            \item a második egyetemi év végéig legalább két félévben kutató- vagy projektmunkát kell végezniük egy felsőbb éves mentor vagy témavezető támogatásával.
			            \item a harmadik és az utána következő egyetemi éveikben kutató- vagy projektmunkát kell végezniük.
			            \item félév végi írásbeli beszámolóban röviden ismertetniük kell a félévben végzett szakmai, tudományos és tanulmányi tevékenységeiket, illetve a jövőbeli terveiket.
		            \end{enumerate}
		      \item Minden műhelytagnak a harmadik egyetemi év végéig \emph{Tudományos Diákköri Konferencián} részt kell vennie.
	      \end{enumerate}
	\item A műhelytagoknak törekedniük kell
	      \begin{enumerate}
		      \item \emph{Tudományos Diákköri Konferencián} előadás bemutatására;
		      \item a \emph{Nemzeti Felsőoktatási Ösztöndíj} és az \emph{Egyetemi Kutatói Ösztöndíj Program} ösztöndíjának elnyerésére;
		      \item tudományos/szakmai pályázatokban való részvételre;
		      \item konferenciákon való részvételre, publikálásra;
		      \item szakmai versenyeken való eredményes szereplésre;
		      \item kari demonstrátori tevékenység folytatására;
		      \item collegiumi óratartásra, óraszervezésre;
		      \item a Collegium szakmai programjain, konferenciáin való részvételre, mind hallgatóként, mind előadóként;
		      \item igény esetén alsóbb évesek mentorálására;
		      \item a Collegium informatikai rendszerének fejlesztésére.
	      \end{enumerate}
	\item A műhely mentorrendszerrel segíti az első és második egyetemi éves műhelytagok szakmai előrehaladását és a collegiumi életbe való becsatlakozását.
	      \begin{enumerate}
		      \item Az őszi félév első műhelygyűlésén minden egyetemi első- és másodéves hallgatóhoz rendelni kell egy-egy felsőbb éves műhelytagot, aki a mentora lesz.
		      \item Egy mentornak legfeljebb két mentoráltja lehet.
		      \item A mentorok figyelemmel kísérik a mentoráltjuk szakmai előrehaladását, az első collegiumi évük után segítik a számukra megfelelő kutatási téma vagy projekt megválasztását.
		      \item A mentoráltak kéthetente beszámolnak tanulmányi előrehaladásukról a mentoruknak.
		      \item A mentorok a félév végén beszámolnak a műhelyvezetőnek a mentoráltjaik eredményeiről.
	      \end{enumerate}
\end{enumerate}

\section{A műhelygyűlés}
\label{par:muhelygyules}

\begin{enumerate}
	\item A műhelygyűlés célja a Műhely vezetője, titkára és tagjai közötti egyeztetésnek keretet adni.
	\item Félévnyitó műhelygyűlésre kerül sor a regisztrációs időszakban vagy a szorgalmi időszak első vagy második hetében. Célja az újonnan felvett hallgatók megismerése, valamint a félév programjának megbeszélése. Az őszi félévnyitó műhelygyűlésen műhelytitkárt kell választani.
	\item Félévzáró műhelygyűlésre a szorgalmi időszak utolsó előtti vagy utolsó hetében vagy a vizsgaidőszakban kerül sor. Célja a félév munkájának összefoglalása, a következő félév megtervezése.
	\item A műhelygyűlésen a Műhely tagjai kötelesek részt venni. Távolmaradás csak indokolt esetben, a műhelyvezetővel egyeztetve történhet.
	\item A Műhely vezetőjének, valamely tagjának vagy az igazgatónak kérésére rendkívüli műhelygyűlést kell összehívni a kérés beérkezését következő 14 napon belül. \label{bek:rendkivuliMuhelygyules}
\end{enumerate}

\section{A műhelytitkár}

\begin{enumerate}
	\item A műhelytitkár a Műhely tagjai által műhelygyűlésen a következő őszi félévnyitó műhelygyűlésig megválasztott tisztségviselő. A műhelytitkárt jelölés után a jelöltek közül a Műhely tagjai egyszerű többséggel, titkos szavazás útján választják. Műhelytitkárnak a Műhely felsőbb éves aktív collegista tagjai jelölhetőek.
	\item A műhelytitkár:
	      \begin{enumerate}
		      \item kapcsolatot tart a Műhely vezetője és tagjai, a Collegium tisztségviselői és szervei, valamint más intézmények között;
		      \item képviseli a Műhelyt a Collegium e célra szervezett fórumain;
		      \item szervezi, illetve koordinálja a szervezését a Műhely rendezvényeinek és tevékenységeinek;
		      \item az aktuális pályázati lehetőségeket figyelemmel követi, továbbítja a műhelytagok felé;
		      \item jelzi a Műhelyben fellépő belső problémákat a műhelyvezető felé, a Műhely vezetőjével kapcsolatos problémákat pedig a Választmány szakmai alelnöke felé.
	      \end{enumerate}
	\item A műhelytitkárt tisztségéből a műhelygyűlés visszahívhatja. A műhelytitkár tisztsége megszűnik műhelytagsága megszűnésével, mandátuma lejártával, illetve visszahívása vagy lemondása esetén.
	\item A Műhelynek lehetősége van műhelytitkár-helyettes választására:
	      \begin{enumerate}
		      \item a műhelytitkár-helyettes mandátumára, mandátumának megszűnésére és jogállására a műhelytitkárra vonatkozó szabályok irányadóak;
		      \item a műhelytitkár-helyettes feladata a segítségnyújtás a műhelytitkár részére;
		      \item a műhelytitkár akadályoztatása esetén a műhelytitkár-helyettes ellátja a műhelytitkár feladatait;
		      \item a műhelytitkár-helyettes a műhelytitkár tisztségének megszűnése esetén a következő műhelygyűlésig ideiglenes műhelytitkárként átveszi a műhelytitkár összes feladatát.
	      \end{enumerate}
	\item Amennyiben a Műhely nem rendelkezik műhelytitkárral és ideiglenes műhelytitkárral sem, \aref{par:muhelygyules}. § (\ref{bek:rendkivuliMuhelygyules}) bekezdésben foglalt időtartamon belül műhelygyűlésen műhelytitkárt kell választani.
\end{enumerate}

\section{Záró rendelkezések}

\begin{enumerate}
	\item Jelen szabályzat megfogalmazására a CTSzK 4. § (8) bekezdése alapján kapott felhatalmazást a Műhely.
	\item Jelen Szabályzat új műhelyszabályzat elfogadásáig érvényes.
	      \begin{enumerate}
		      \item Új műhelyszabályzat megalkotását a Műhely vezetője, a műhelytitkár, valamint a Műhely tagjai kezdeményezhetik. A műhelyszabályzat és módosítások elfogadására műhelygyűlésen kerülhet sor, a Műhely vezetőjét, titkárát és tagjait is meghallgató érdemi vitát követően egyhangú szavazás útján.
		      \item A Szabályzat elfogadására a CTSzK 4. § (8) bekezdése alapján, a műhelygyűlés javaslatára, a Collegium igazgatója jogosult.
	      \end{enumerate}
	\item A műhelyszabályzatban nem szabályozott kérdések tekintetében a CTSzK vagy az ELTE Hallgatói Követelményrendszer az irányadó.
	      \begin{enumerate}
		      \item A vitás kérdések elbírálására a Collegium igazgatója és a Műhely vezetője közötti megbeszélésen kerül sor.
	      \end{enumerate}
\end{enumerate}

\vspace{0.3in}
Budapest, 2024. május 21.
\vspace{0.6in}

\begin{minipage}{3in}
	\begin{center}
		Dr. Lócsi Levente\\
		műhelyvezető
	\end{center}
\end{minipage}
\begin{minipage}{3in}
	\begin{center}
		Dr. Horváth László\\
		igazgató
	\end{center}
\end{minipage}

\end{document}
