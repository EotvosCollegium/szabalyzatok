\documentclass{../styles/rulebook}

\begin{document}
\section*{ELTE Eötvös József Collegium \\ Informatikai Műhely\\ \vspace{0.5em} Műhelyszabályzat} 

\vspace{2em}

\section*{Preambulum}

Az Informatikai Műhely (a továbbiakban: Műhely) az Eötvös Loránd Tudományegyetem (a továbbiakban: ELTE) Eötvös József Collegiumának (a továbbiakban: Collegium) szakmai és hallgatói közössége.
A Collegium nagymúltú hagyományainak, célkitűzéseinek megfelelően, a 2004-ben alapított Műhely a tudós, önálló kutatásaikkal is kitűnő, hivatásukat szerető és szívvel-lélekkel végző informatikusok képzését, tehetséggondozását tekinti feladatának. A Műhely annak a felismerésnek a szellemében fogant, miszerint a XXI. századi informatikus képzésben feltétlenül szükséges a legújabb elméleti és gyakorlati eredmények megismerése és a modern informatika egységes, átfogó ismerete.

\section{Célkitűzések}

\begin{enumerate}
	\item A Műhely elsődleges célja az, hogy segítse az ELTE Informatikai Kar legtehetségesebb hallgatóinak szakmai és tudományos előmenetelét, melynek legfőbb eszközei:
	\begin{enumerate}
		\item műhelykurzusok meghirdetésével a műhelytagok eredeti képzésének kiegészítése;
		\item kutatószeminárium szervezése, ahol a Műhelytagok megismerhetik egymás kutatásait, és támogatást kaphatnak sajátjukhoz.
	\end{enumerate}
	\item Összetartó, együttműködő közösség kialakítása.
\end{enumerate}

\section{A Műhely tagjai}

\begin{enumerate}
	\item A Műhelybe az ELTE Informatikai Karának hallgatói nyerhetnek felvételt.
	\begin{enumerate}
		\item A Műhely vezetőjének engedélyével az ELTE egyéb hallgatói is felvételt nyerhetnek.
	\end{enumerate}
	\item A műhelytagok felvételének részleteiről a Collegium felsőbb szabályzatai rendelkeznek az alábbi kiegészítésekkel:
	\begin{enumerate}
		\item a szakmai bizottságnak a Műhely vezetője (vagy megbízottja) mellett tagja a műhelytitkár, a további tagokat a Műhely vezetője kéri fel;
		\item a Collegium általános felvételi eljárását a Műhely egy írásbeli, szakmai feladatsorral egészíti ki, melynek megírására a szakmai elbeszélgetések előtt kerül sor, és eredménye elsősorban a szakmai bizottság munkáját segíti.
	\end{enumerate}
	\item A műhelytagság megszűnik:
	\begin{enumerate}
		\item a collegiumi státusz megszűnésével automatikusan;
		\item a műhelytag a Műhely vezetőjének címzett írásos lemondásával;
		\item a műhelyszabályzatban foglalt követelmények nem teljesítése esetén a Műhely vezetőjének döntése alapján.
	\end{enumerate}
	\item A műhelytagoknak törekedniük kell a Collegium és a Műhely közösségi életében való aktív részvételre.
\end{enumerate}

\section{A Műhely tanulmányi rendje}

\begin{enumerate}
\item A Műhely tanulmányi rendjével törekszik a műhelytagok kiemelkedő szakmai és tudományos tevékenységét segíteni és ösztönözni.
\item A műhelytagok tanulmányi és szakmai kötelességei:
	\begin{enumerate}
		\item A Műhely által szervezett kurzusok közül legalább egyet minden félévben teljesíteni kell.
		\item A műhelytagoknak
			\begin{enumerate}
				\item az első félévükben 
					\begin{itemize}
						\item a műhely által számukra felkínált kurzusok közül kell legalább egyet teljesíteniük;
						\item pályázniuk kell a kari \emph{Neumann János Tehetséggondozó Kör}be.
					\end{itemize}
				\item a második félévükben
					\begin{itemize}
						\item a Modern elméletek az informatikában I. tárgyat kell elvégezniük.
					\end{itemize}
				\item a harmadik félévükben 
					\begin{itemize}
						\item a Modern elméletek az informatikában II. tárgyat kell elvégezniük;
						\item részt kell venniük a Műhely által meghirdetett kutatószemináriumon.
					\end{itemize}
				\item a negyedik és az utána következő félévjeikben 
					\begin{itemize}
						\item részt kell venniük a Műhely által meghirdetett kutatószemináriumon, melyen előadást is kell tartaniuk;
						\item kutató vagy projekt munkát kell minden félév során végezniük. Erről minden félévnyitó műhelygyűlésig a Műhely vezetőjének a témavezető által aláírt igazolást szükséges benyújtani.
					\end{itemize}
			\end{enumerate}
		\item Minden műhelytagnak a harmadik év végéig \emph{Tudományos Diákköri Konferencián} kell részt vennie.
	\end{enumerate}
\item A műhelytagoknak törekedniük kell
	\begin{enumerate}
		\item \emph{Nemzeti Felsőoktatási Ösztöndíj} elnyerésére;
		\item tudományos/szakmai pályázatokban való részvételre;
		\item konferenciákon való részvételre, publikálásra;
		\item szakmai versenyeken való eredményes szereplésre;
		\item collegiumi óratartásra, óraszervezésre, kari demonstrátori tevékenység folytatására, különös tekintettel a Műhely által szervezett kurzusokra a Collegium más műhelyeinek tagjai, illetve a Műhely elsőévesei számára.
		\item a Collegium szakmai programjain, konferenciáin való részvételre, mind hallgatóként, mind előadóként.
	\end{enumerate}
\item A műhely mentorrendszerrel segíti az elsőéves hallgatókat a tanulmányaikban, illetve collegiumi beilleszkedésükben.
	\begin{enumerate}
		\item A tavaszi félév első műhelygyűlésén minden elsőéves hallgatóhoz rendelni kell egy felsőbbéves műhelytagot, aki a mentora lesz.
		\item A mentorok a mentoráltjukat
			\begin{itemize} 
				\item aktívan hívják a collegiumi közösségi rendezvényekre;
				\item segítik az adminisztratív teendőkben;
				\item igény esetén a tanulmányaikban segítik (közvetlen vagy közvetett módon).
			\end{itemize}
		\item A mentorok figyelemmel kísérik a mentoráltjuk tanulmányi előrehaladását, az első collegiumi évük után segítik a számukra megfelelő kutatási téma megválasztásában.
	\end{enumerate}
\end{enumerate}

\section{A műhelygyűlés}

\begin{enumerate}
	\item A műhelygyűlés célja a Műhely vezetője, titkára és tagjai közötti egyeztetésnek keretet adni.
	\item Félévnyitó műhelygyűlésre kerül sor a félév első vagy második hetén. Célja az újonnan felvett hallgatók megismerése, az előző félév munkájának összefoglalása, valamint a jövő félév programjának megbeszélése.
	\item A műhelygyűlésen a Műhely tagjai kötelesek részt venni.
	\item A műhelytagok a műhelygyűlésen kötelesek röviden beszámolni az előző műhelygyűlés óta eltelt időszakban végzett szakmai, tudományos és tanulmányi tevékenységükről, illetve a jövőbeli terveikről.
	\item A Műhely vezetője jogosult rendkívüli műhelygyűlés összehívására.
	\item A műhelygyűlés jelöli ki az elsőéves hallgatók számára kitűzött kurzuskínálatot.
\end{enumerate}

\section{A műhelytitkár}
\begin{enumerate}
	\item A műhelytitkár a következő feladatokat látja el:
	\begin{enumerate}
		\item kapcsolattartás a Műhely vezetője és tagjai, a Collegium tisztségviselői és szervei, valamint más intézmények között;
		\item a műhelytagok szakmai és tanulmányi eredményeinek követése;
		\item a Műhely képviselete a Collegium e célra szervezett fórumain, elsődlegesen a Tudományos Bizottság kiterjesztett ülésein;
		\item a műhelygyűléseken jegyzőkönyvek készítése, valamint e jegyzőkönyv továbbítása a Műhely vezetője, és tagjai részére;
		\item a Műhely rendezvényeinek és tevékenységeinek megszervezése, illetőleg a szervezés koordinációja;
		\item a Műhely féléves beszámolójának megírása;
		\item az aktuális pályázati lehetőségek figyelemmel követése, továbbítása a műhelytagok felé.
	\end{enumerate}
	\item Műhelytitkár a Műhely felsőéves tagja lehet.
	\item A műhelytitkárt valamennyi őszi félévnyitó műhelygyűlésen a Műhely tagjai jelölés után a jelöltek közül egyszerű többséggel választják.
	\begin{enumerate}
	    \item  A műhelytitkár kinevezése rendesen egy évre szól.
		\item A műhelytitkárt azonnali hatállyal felmentheti pozíciójából a Műhely vezetője, amennyiben megítélése szerint a rá bízott feladatait képtelen ellátni.
		\item A műhelytitkárnak joga van felmentését kérni a Műhely vezetőjétől, amennyiben javaslatot tesz számára a következő műhelytitkár személyére.
	\end{enumerate}
	\item A műhelytitkár törekedjen arra, hogy a Műhely tagsága szervezzen az informatikát népszerűsítő és/vagy alapszintű informatika tudást átadó kurzusokat, foglalkozásokat más műhelyek tagjainak részére, illetve középiskolások számára (pl. kari nyílt napokon, kutatók éjszakáján).
\end{enumerate}

\section{Zárórendelkezések}

\begin{enumerate}
	\item Jelen szabályzat megfogalmazására a CTSzK 4. § (8) bekezdése alapján kapott felhatalmazást a Műhely.
	\item Jelen Szabályzat új műhelyszabályzat elfogadásáig érvényes.
	\begin{enumerate}
		\item Új műhelyszabályzat megalkotását a Műhely vezetője, a műhelytitkár, valamint a Műhely tagjai kezdeményezhetik. A műhelyszabályzat és módosítások elfogadására a félévnyitó műhelygyűlésen kerülhet sor, a Műhely vezetőjét, titkárát és tagjait is meghallgató érdemi vitát követően egyhangú szavazás útján.
		\item A Szabályzat elfogadására a CTSzK 4. § (8) bekezdése alapján, a műhelygyűlés javaslatára, a Collegium igazgatója jogosult.
	\end{enumerate}
	\item A műhelyszabályzatban nem szabályozott kérdések tekintetében a CTSzK vagy az ELTE Hallgatói Követelményrendszer az irányadó.
	\begin{enumerate}
		\item A vitás kérdések elbírálására a Collegium igazgatója és a Műhely vezetője közötti megbeszélésen kerül sor.
	\end{enumerate}
\end{enumerate}

\vspace{0.3in}
Budapest, 2021. május 20.
\vspace{0.6in}

\begin{minipage}{3in}
\begin{center}
Kozsik Tamás\\
műhelyvezető
\end{center}
\end{minipage}
\begin{minipage}{3in}
\begin{center}
Dr. Horváth László\\
igazgató
\end{center}
\end{minipage}

\end{document}