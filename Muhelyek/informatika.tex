\documentclass{rulebook}

\begin{document}
\section*{ELTE Eötvös József Collegium \\ Informatika műhely\\ \vspace{0.5em} Műhelyszabályzat} 

\vspace{2em}

\section*{Preambulum}
TODO


\section{Bevezetés}

\begin{enumerate}
	\item Az Eötvös József Collegium nagymúltú hagyományainak, célkitűzéseinek meg-felelően, a 2004-ben alapított Informatikai Műhely a tudós, önálló kutatásaikkal is kitűnő, hivatásukat szerető és szívvel-lélekkel végző informatikusok képzését tekinti feladatának.
	\item Az Informatikai Műhely annak a felismerésnek a szellemében fogant, miszerint a XXI. századi informatikus képzésben feltétlenül szükséges a legújabb elméleti és gyakorlati eredmények megismerése és a modern informatika egységes, átfogó ismerete.
	\item Az Informatikai Műhely munkájának célja az informatika és az informatikával kapcsolatos szakú collegista hallgatók képzésének kiegészítése. Ebből egyértelműen következik, hogy a collegistáknak többet kell teljesíteniük nem collegista évfolyam-társaikhoz képest. E többletteljesítmény kereteit a Műhely Szabályzata határozza meg.
	\item A collegiumi előadások, gyakorlatok és szemináriumok többségét az ELTE Informatikai Karának tanszékei saját tanegységükként ismerik el.
	\item Az EC és az ENS történelmi kapcsolatai és az ezekben rejlő tanulmányi lehetőségek miatt az Informatikai Műhely tagjainak figyelmébe ajánljuk a francia és a német nyelv tanulását is.
\end{enumerate}


\section{Szabályok}

\begin{enumerate}
	\item Az Informatikai Műhely az informatikai szakterület legtekintélyesebb magyar tudósai közül tiszteletbeli elnököt választhat.
	\item Az Informatikai Műhely tagja a Collegiumban főállásban vagy óraadóként informatikai tanegységet oktató tanár, valamint minden olyan bentlakó vagy bejáró collegista, akinek legalább az egyik szakja az Informatikai Karhoz tartozik.
	\item Az Informatikai Műhely minden tanévben két, szemeszterenként egy-egy mű-helygyűlést tart, amelyek egyikét lehetőség szerint kirándulással köti egybe.
	\item Az Informatikai Műhely collegista tagjai minden évben a tanév első műhely-gyűlésén maguk közül szótöbbséggel titkárt választanak.
	\item Az Informatikai Műhely szemeszterenként megújított előadás, speciális kollégium, ill. szeminárium kínálatát a Műhely vezetője a Műhely tagjainak véleményét meghallgatva alakítja ki.
	\item A collegisták a szemeszterenként meghirdetett előadások, speciális kollégiumok, szemináriumok közül legalább egy, szakjuknak vagy szakjaiknak megfelelő, az ELTE által is meghirdetett tárgyat kötelesek leckekönyvükbe is felvenni és abból levizsgázni. VII. Az Informatikai Műhely által meghirdetett előadásokat, speciális kollégiumokat, szemináriumokat minden collegista felveheti. Az Informatikai Műhely tagjai más Műhely által meghirdetett előadásokat, speciális kollégiumokat, szemináriumokat is felvehetnek, és ezek sikeres elvégzése esetén, a Műhely vezetőjének beleegyezésével a VI. pontban megfogalmazott követelményt kiválthatják.
	\item Az egyes alkalmakon a tanár engedélyével külsős hallgatók is részt vehetnek. A collegiumi órák lehetőség szerint a Collegium épületében legyenek.
	\item A Collegium biztosította tutoriális képzést kihasználva az Informatikai Műhely tagjai egyetemi tanulmányaik alatt, de legkésőbb a nyolcadik félév végéig, legalább egy olyan dolgozatot kötelesek írni, melyet az Eötvös pályázatra és az Informatikai Kar Tudományos Diákköri Konferenciájára is benyújtanak.
	\item Amennyiben a fenti követelményeknek egy bentlakó vagy bejáró collegista nem tesz eleget, a Collegium tagjai közül a Collegium tanári testületének és Igazgatójának egyetértésével kizárható.
\end{enumerate}


\section{Zárórendelkezések}

TODO: műhelyvezető megnevezése?
\begin{enumerate}
	\item A jelen műhelyszabályzat által nem szabályozott kérdésekben elsődlegesen a Collegium Szervezeti és Működési szabályzata, valamint az Oktatási, Tanulmányi Szabályzata és Követelményrendszere, másodsorban az ELTE Hallgatói követelményrendszere az irányadó.
	\item Jelen műhelyszabályzat 2017. január 1-től hatályos.
\end{enumerate}

\end{document}