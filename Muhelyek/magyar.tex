\documentclass{rulebook}

\begin{document}
\section*{ELTE Eötvös József Collegium \\ Magyar műhely\\ \vspace{0.5em} Műhelyszabályzat} 

\vspace{2em}

\section*{Preambulum}
TODO

\section{A Műhely célkitűzése}

\begin{enumerate}
	\item Az ELTE Eötvös József Collegium (a továbbiakban Collegium) Magyar műhelye (a továbbiakban műhely) a Collegium egyik legnagyobb tradícióval rendelkező szakmai közössége. A műhely célja, hogy hagyományának megfelelően – műhelykurzusok és valódi műhelymunka formájában – magas szinten segítse tagjai tudományos előmenetelét és ellássa tagjainak szakmai érdekképviseletét.
\end{enumerate}


\section{A műhely tagjai}

\begin{enumerate}
	\item A műhely tagjai a Collegium magyar alapszakos és osztatlan magyartanár szakos hallgatói mellett a Magyar Irodalom-és Kultúratudományi Intézet, valamint a Magyar Nyelvtudományi és Finnugor Intézet által gondozott valamennyi szak, illetve a az Irodalomtudományi és a Nyelvtudományi Doktori Iskola hallgatói lehetnek.
	\item Eseti elbírálás alapján, a műhelygyűlésen egyszerű többséggel hozott döntés értelmében  magyar minor, illetve más szak hallgatói is felvehetők.
	\item A Collegium bármely más műhelyének tagja kérvényezheti felvételét a műhelyvezetőnél. Erről a  műhelygyűlés dönt.
	\item A műhelytagság automatikusan megszűnik a collegiumi státusz megszűntével, illetve amennyiben a műhelytag erre vonatkozóan írásos kérvényt nyújt be, és ezt a műhelygyűlés elfogadja.
	\item A műhelytagok felvételének módjáról  a Collegium felsőbb szabályzatai rendelkeznek.
\end{enumerate}


\section{A műhelygyűlés}

\begin{enumerate}
	\item A műhelygyűlés a műhely legfőbb határozati szerve.
	\item A műhelygyűlésen szavazati joggal a műhelyvezető, a két kabinetvezető és a műhelytagok rendelkeznek.
	\item A műhelygyűlés határozatképes, ha a műhelytagok legalább fele és a műhelyvezető (vagy a műhely valamely kabinetjének vezetője) jelen van. A műhelygyűlés határozatait a jelenlévők többségi szavazatával hozza.
	\item A műhelygyűlést a műhelyvezető (vagy a műhely valamely kabinetjének vezetője vagy megbízatása esetén a műhelytitkár) hívja össze írásban, legkésőbb 7 nappal a tervezett időpont előtt, egyúttal közrebocsátja a tervezett napirendet, valamint a határozathozatalhoz szükséges	írásbeli anyagokat. A napirendet bármelyik műhelytag vagy kabinetvezető javaslatára módosíthatja a műhelygyűlés.
	\item Ha az összehívott műhelygyűlés határozatképtelennek bizonyul, a műhelyvezető (vagy a műhely valamely kabinetjének vezetője, vagy megbízatása esetén a műhelytitkár) köteles 8 napon belül pótgyűlést összehívni. A pótgyűlés a megjelent műhelytagok számától függetlenül határozatképes.
	\item A műhelygyűlés feladatai közé tartozik a műhelyszabályzat és a curriculum elfogadása, és szükség esetén módosítása is. E kettő csak a műhelytagság legalább kétharmadának hozzájárulásával történhet.
	\item Ugyancsak a műhelygyűlés feladata a műhelyhonlap felelősének kijelölése, és a műhelykonferenciák témájának, helyszínének, időpontjának és az előadáscímek leadási határidejének a meghatározása.
	\item A félév elején és a félév végén a műhely műhelygyűlést tart. A félév eleji műhelygyűlés feladata megszervezni az adott félévet, megerősíteni vagy leváltani a műhelytitkárt, illetve megtervezni a következő félév menetét. A félév végi műhelygyűlés feladata az elmúlt félév értékelése, a tagok teljesítményének értékelése (különös tekintettel a műhely által támasztott követelményekre), továbbá a következő félévi kurzusokra tett javaslatok megvitatása.
	\item A műhelygyűlésről a műhelytitkár (akadályoztatása esetén az általa kijelölt műhelytag) jegyzőkönyvet köteles készíteni, és azt a műhelygyűlést követő hét napon belül a műhelytagok rendelkezésére bocsátani.
\end{enumerate}


\section{A műhely tagjainak további kötelezettségei}

\begin{enumerate}
	\item A műhelykurzusok meghirdetésének és látogatásának feltételeiről az ELTE Hallgatói követelményrendszer 27., és 55. §-a rendelkezik.
	\item A műhely valamennyi felsőbb éves tagja köteles minden tanévben az alábbi tevékenységek legalább egyikét teljesíteni:
	\begin{enumerate}
		\item	a kétévente megrendezett műhelykonferencián kívül legalább egy konferencián előadni,
		\item	a Tudományos Ösztöndíj pályázatra pályázatot benyújtani,
		\item	a kari Tudományos Diákköri Konferencián részt venni,
		\item	tudományos folyóiratban vagy gyűjteményes kötetben publikációt közölni,
		\item	tudományszervezési feladatokat ellátni.
	\end{enumerate}
	\item A tavaszi félévzáró műhelygyűlés előtt minden műhelytag köteles a (2) pontban meghatározott tevékenységek elvégzéséről írásbeli beszámolót küldeni. 
	\begin{enumerate}
		\item 	A beszámolókat a műhely- és a kabinetvezetőkön túl valamennyi műhelytag megkapja, majd a műhelytitkár ellenőrzi, hogy a beszámolók megfelelnek-e a (2) pontban foglaltaknak.
		\item  Ha a beszámoló nem felel meg a (2) pont elvárásainak, a műhelytitkár javaslatot tesz a műhelygyűlésnek a beszámoló megvitatására. Hasonló javaslattal a műhelyvezető, valamint a kabinetvezetők is élhetnek.
		\item A beszámoló elutasítása esetén, amely a műhelygyűlés egyszerű többségű határozata alapján történik, a műhelyvezető a soron következő Tanári Értekezleten javaslatot tesz a műhelytag Collegiumból való elbocsátására.
	\end{enumerate}
	\item A műhely felsőbb éves tagjai kötelesek előadni a kétévente megrendezett műhelykonferencián. \item A műhely senior tagjait a műhelygyűlés tanévenként egy szemeszterre felkérheti a műhely tagjai számára indított kurzus megtartására.
	\item A műhely elsőéves tagjai kötelesek a műhely által e célra szervezett konferencián szerepelni.  Az erre való felkészülésben őket a műhely egy felsőbb éves tagja (a továbbiakban tutor) segíti.
	\item Tutor a műhely bármely, az egyetemi képzésben legalább négy  félévet teljesített tagja lehet. 
	\begin{enumerate}
		\item A műhely minden, legalább az alapszak vagy osztatlan tanári szak második évfolyamát végző tagja köteles  a  tavaszi  félévet  lezáró  műhelygyűlésen  jelezni,  hogy  felkérése  esetén  vállal-e  tutori feladatot, és a kutatás lehetséges témáját megjelölni, amelynek feltárásában vállalja, hogy az elsőéves hallgatót segíti.
		\item 	 A tutort az elsőéves hallgató felkérheti más tematikájú kutatás vezetésére is, ezt azonban a tutor nem köteles elvállalni.
	\end{enumerate}
	\item A tutor a tutorálás ideje alatt legalább kéthavonta köteles a tutorálás előrehaladásáról a műhelyvezetőnek írásban beszámolni.
	\item Amennyiben nem jelentkezik legalább az elsőéves hallgatók számával megegyező számú tutor, akkor a műhelyvezető feladata további műhelytagok kijelölése a tutori pozícióra.
	\item A félév eleji műhelygyűlésen bemutatkoznak azok a műhelytagok, akik tutorálást vállalnak, és megnevezik a lehetséges kutatás témáját.
	\item A műhely elsőéves tagjai – miután a tutorral, a műhelytitkárral és a műhelyvezetővel (vagy a kutatási területnek megfelelő kabinet vezetőjével) előzetesen egyeztettek – október végéig nevezik meg a választott kutatási területüket és tutorukat.
	\item Az 5. §-ban meghatározott követelmények megszegése a műhelyből való kizárást vonja maga után. 
	\begin{enumerate}
		\item A műhely követelményei alól személyes és eseti elbírálás alapján a műhelygyűlés adhat előzetes felmentést.
	\end{enumerate}
\end{enumerate}


\section{Tisztségek}

\begin{enumerate}
	\item A műhely élén a műhelyvezető, az irodalmi és a nyelvészeti kabinet vezetői állnak. Tevékenységükről és megválasztásukról a Collegium Szervezeti és Működési szabályzata, valamint az Oktatási, Tanulmányi Szabályzat és Követelményrendszer rendelkeznek.
	\item A műhelyvezető és a műhely közötti kommunikációt, a műhely működésének koordinálását a műhelytitkár végzi. Indokolt esetben több műhelytitkár is kinevezhető, ebben az esetben tisztségük megnevezése: társműhelytitkár.
	\item A műhelytitkár további feladatai:
	\begin{enumerate}
		\item	A műhely érdekeinek képviselete, és együttműködés biztosítása a Választmány, az Igazgatóság, a Kuratórium, az egyetemen működő Tudományos Diákkörök és más collegiumi és Collegiumon kívüli műhelyek és tudományos fórumok irányában.
		\item	A műhely munkáját összefogni, a műhelykurzust tartó oktatókkal  kapcsolatot tartani.
		\item	A műhely számára legalább félévente szakmai és nem szakmai programokat szervezni. Ehhez tartozik a műhely munkájának anyagi feltételeit megteremtő pályázatok keresése, a pályázati anyag összeállítása, a pályázatok lebonyolítása.
		\item	A collegiumi felvételi diákbizottsági meghallgatására műhelytagokat delegálni.
		\item	A kétévente megrendezendő műhelykonferenciát és az elsőéves műhelytagok számára rendezett konferenciát megszervezni.
		\item	A műhely adminisztrációjának vitele, a műhelynévsor rendszeres frissítése.
	\end{enumerate}

\end{enumerate}


\section{Zárórendelkezések}

\begin{enumerate}
	\item A jelen műhelyszabályzat által nem szabályozott kérdésekben elsődlegesen a Collegium   Szervezeti és Működési szabályzata, valamint az Oktatási, Tanulmányi Szabályzata és Követelményrendszere, másodsorban  az  ELTE  Hallgatói követelményrendszere  az  irányadó.  
	\item Jelen műhelyszabályzat mellékletét képezi a Curriculum az Eötvös Collegium Magyar műhelye számára című dokumentum.
	\item Jelen műhelyszabályzat 2017. január 1-től hatályos.
\end{enumerate}

\end{document}