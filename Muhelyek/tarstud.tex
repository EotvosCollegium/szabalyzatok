\documentclass{rulebook}

\begin{document}
\section*{ELTE Eötvös József Collegium \\ Társadalomtudományi műhely\\ \vspace{0.5em} Műhelyszabályzat} 

\vspace{2em}

\section*{Preambulum}
A Társadalomtudományi Műhely (a továbbiakban: Műhely) az Eötvös Loránd Tudományegyetem
(a továbbiakban: ELTE) Eötvös József Collegiumának (a továbbiakban:
Collegium) szakmai és hallgatói közössége. A Műhely létéről a Collegium Oktatási,
Tanulmányi Szabályzata és Követelményrendszere (a továbbiakban: CTSzK) 4. § (7)
p) bekezdése hozza létre. %TODO nem feltétlen ez hozza létre


\section{Általános célkitűzések}

\begin{enumerate}
	\item A Műhely elsődleges célja az, hogy segítse az ELTE Társadalomtudományi Kar legtehetségesebb
	hallgatóinak szakmai és tudományos előmenetelét. E cél elérésére elsődlegesen
	a CTSzK 4. § (3) alapján meghirdethető műhelykurzusok, ezenfelül készségfejlesztő
	programok és személyre szabott tutorálás segítségével törekszik.
	\item A Műhely céljai továbbá:
	\begin{enumerate}
		\item konferencia- és publikálási lehetőség biztosítása a Műhely hallgatóinak;
		\item a tevékenysége iránt érdeklődő középiskolás hallgatók megszólítása;
		\item a Collegium és a Műhely népszerűsítése a magyarországi és határon túli magyar
		nyelvű felsőoktatásban.
	\end{enumerate}
\end{enumerate}


\section{Felvételi eljárás}

\begin{enumerate}
	\item A felvételi eljárás collegiumi szintű szabályairól a CTSzK 2. §-a rendelkezik. 
	\item A Műhelybe az ELTE Társadalomtudományi Karának hallgatói nyerhetnek felvételt.
	\item A Kar azon aktív jogviszonnyal rendelkező hallgatója jelentkezhet a Műhelybe, aki tanulmányait első- vagy másodéves alapképzésben, elsőéves mesterképzésben vagy doktori képzésben kezdi meg az adott felvételi időszak idején.
	\begin{enumerate}
		\item A Műhely vezetőjének engedélyével az ELTE Bölcsészettudományi Kar releváns szakjainak aktív jogviszonnyal rendelkező hallgatók is felvételt nyerhetnek.
		\item A Collegium igazgatójának és a Műhely vezetőjének egyetértésével az Egyetem már karain aktív jogviszonnyal rendelkező hallgató is felvételt nyerhet.
	\end{enumerate}
	\item A Műhely szakmai felvételi bizottságának állandó tagjai a Műhely vezetője, a műhelytitkár.
	\begin{enumerate}
		\item A Műhely vezetője a CTSzK 2. § (12) értelmében jogosult távollétében jogait egy előzetesen megjelölt szaktanárra átruházni.
		\item A Műhely vezetője engedélyezheti a Műhely felsőbb éves tagjainak részvételét is a szakmai felvételi bizottságban.
		\item Ha a Műhelynek van vezető tanára, ő is állandó tagja a szakmai felvételi bizottságnak.
	\end{enumerate}
\end{enumerate}


\section{A Műhely tanulmányi rendje}

\begin{enumerate}
	\item A Műhely törekszik a hallgatók sokféleségének és érdeklődési körének megfelelő
	kurzuskínálat kialakítására.
	\item A Műhely kurzusaira a Műhely tagjai iratkozhatnak be.
	\begin{enumerate}
		\item Az adott műhelykurzus oktatója a Műhely vezetőjének engedélyével és a műhelytitkárral történt konzultálást követően, egyéni elbírálás alapján a Collegium más Műhelyei tagjainak engedélyezheti az óra felvételét. Ilyen esetben azt kell mérlegelni, hogy a hallgató számára mennyire indokolt a kurzus felvétele, illetve megmarad-e az óra szeminarizált jellege.
	\end{enumerate}
	\item A Műhely tagja köteles valamennyi félévben:
	\begin{enumerate}
		\item részt venni a Műhely által valamennyi félévben meghirdetett kutatószemináriumon; az óra célja a készségfejlesztés, kutatási- és előadási kompetenciák fejlesztése és egymás kutatási területeinek jobb megismerése.
		\item a Műhely által szervezett kurzusok közül legalább egy felvételére és teljesítésére a kutatószemináriumon felül.
		\begin{enumerate}
			\item A műhelytitkár a Műhely vezetőjének egyetértése után ajánlást tesz, hogy a Műhely tagjainak mely kurzusokat érdemes az adott félévben elvégeznie.
			\item Az egyes kurzusok teljesítésének módjáról az óraadó tanárok döntenek, a Műhely vezetőjének előzetes beleegyezésével.
			\item A Műhely tagja csak a Műhely vezetőjével, a műhelytitkárral és az adott kurzus oktatójával történt előzetes konzultáció után, nyomós indok miatt adhatja le az órát, vagy dönthet annak fel nem vételéről.
			\item Műhelyórát tarthat a Műhely vezetőjének beleegyezésével a Műhely bármely tagja.
			\item A Műhely törekszik a magyar nyelvű kurzusok mellett idegen (elsősorban angol) nyelvű órák szervezésére is.
		\end{enumerate}
		\item a félévnyitó műhelygyűlésen meghatározott időpontra benyújtani egy társadalomtudományi relevanciával bíró, tudományos cikk rezüméjét, felsőbb éves tagoknak recenziót. A rezümét (recenziót) egy tetszőleges, 15-20 oldalas, idegen nyelvű cikkből kell megírni magyar nyelven, majd a műhelytitkáron keresztül a Műhely vezetőjének eljuttatni.
		\begin{enumerate}
			\item A rezümé (recenzió) értékelésére a félévközi műhelygyűlésen kerül sor.
		\end{enumerate}
		\item a vizsgaidőszak végéig benyújtani egy társadalomtudományi témában írt dolgozatot, amelyet a műhelytitkáron keresztül juttat el a Műhely vezetőjéhez.
		\begin{enumerate}
			\item A Műhely vezetőjének előzetes engedélyével a dolgozat részben egy adott félévi szemináriumi dolgozat anyagára is épülhet.
			\item A műhelydolgozat megvitatására a félévnyitó műhelygyűlésen kerül sor.
		\end{enumerate}
		\item a Műhely programjainak szervezésében részt venni.
		\item szakmai előmeneteléről folyamatosan tájékoztatni a Műhely vezetőjét a műhelytitkáron keresztül (elnyert díjak, pályázatok, versenyeredmények).
		\item felsőbb éves hallgatóként segíteni az újonnan felvett műhelytagokat szakmai fejlődésükben, a Collegiumba való beilleszkedésükben.
	\end{enumerate}
\end{enumerate}


\section{A tagság megszűnése}

\begin{enumerate}
	\item A műhelytagság megszűnik a jelen Szabályzat 3. § (3) pontban foglalt valamely feltétel vagy feltételek sorozatos nem teljesítése által.
	\begin{enumerate}
		\item E feltételek teljesítése alól a Műhely vezetőjének előzetes engedélye esetén mentesülhet a tag.
		\item A 4. § (1) a. pontban szereplő előzetes engedély nem eredményezheti a műhelytagsággal járó kötelezettségek kiüresítését.
	\end{enumerate}
	\item A műhelytag jogviszonyának esetleges megszűnését a CTSzK 7. § (4)--(8) bekezdése szabályozza.
\end{enumerate}


\section{A műhelygyűlés}

\begin{enumerate}
	\item A műhelygyűlés célja a Műhely vezetője, vezető tanára, titkára és tagjai közötti egyeztetésnek keretet adni.
	\item Egy félévben minimálisan három műhelygyűlés megtartására kerül sor.
	\begin{enumerate}
		\item A félévnyitó műhelygyűlésre a félév első vagy második hetében kerül sor. Célja az újonnan felvett hallgatók megismerése, az előző félév munkájának összefoglalása, valamint a jövő félév programjának megbeszélése.
		\begin{enumerate}
			\item A félévnyitó műhelygyűlés során a Műhely webfejlesztőt jelöl és választ, aki a Műhely honlapjának és internetes megjelenéseinek frissítéséért felelős.
		\end{enumerate}
		\item A félévközi műhelygyűlésre a félév közben kerül sor, a félévnyitó műhelygyűlésen megbeszéltek alapján. E műhelygyűlés célja a félév addigi eredményeinek áttekintése, valamint a rezümék (recenziók) értékelése.
		\item A félévzáró műhelygyűlésre a szorgalmi időszak végén kerül sor. Célja az elmúlt félév áttekintése, a jövőbeli programok felvázolása.
	\end{enumerate}
	\item A Műhely vezetője jogosult a műhelygyűlés időpontjának megváltoztatására.
	\item A műhelygyűlésen a műhelytitkár és a Műhely tagja köteles részt venni.
	\begin{enumerate}
		\item A Műhely vezetője előzetes engedélyével maximálisan kettő műhelytag távolmaradásával is lehetséges megtartani a műhelygyűlést. Amennyiben több hallgató előzetesen jelzi, hogy a megbeszélt időpontban nyomós okból nem tudna megjelenni, a műhelytitkár köteles új időpontot javasolni a Műhely vezetője számára.
		\item A műhelytitkár felel a műhelygyűlésen elhangzottak lejegyzéséért és annak a résztvevőkhöz történő eljuttatásáért.
	\end{enumerate}
	\item A Műhely vezetője jogosult rendkívüli műhelygyűlés összehívására.
\end{enumerate}


\section{A műhelytitkár}

\begin{enumerate}
	\item A műhelytitkár a következő feladatokat látja el:
	\begin{enumerate}
		\item kapcsolattartás a Műhely vezetője és tagjai, a Collegium tisztségviselői és szervei, valamint más intézmények között.
		\item a Műhely képviselete a Collegium e célra szervezett fórumain, elsődlegesen a Tudományos Bizottság kiterjesztett ülésein.
		\item a műhelygyűléseken jegyzőkönyv készítése, valamint e jegyzőkönyv továbbítása a Műhely vezetője,és tagjai részére.
		\item a Műhely rendezvényeinek és tevékenységeinek megszervezése, illetőleg a szervezés koordinációja.
		\item a rezümék (recenziók) és műhelydolgozatok határidőre történő összegyűjtése.
		\item a Műhely féléves beszámolójának megírása.
	\end{enumerate}
	\item A műhelytitkár a 6. § (1) pontban jelzett tevékenységeit a Műhely vezetője részéről kapott utasítások alapján látja el. A műhelytitkár folyamatosan tájékoztatja Őt az általa végzett tevékenységről.
	\begin{enumerate}
		\item A Műhely egésze képviseletekor a műhelytitkár köteles előzetesen konzultálni a Műhely tagjaival is.
	\end{enumerate}
	\item A műhelytitkár a Műhely felsőbb éves hallgatója, vagy a Műhely már végzett, de még aktív egyetemi hallgatói jogviszonnyal rendelkező hallgatója lehet.
	\item A új műhelytitkárt valamennyi őszi félévnyitó műhelygyűlésen a Műhely vezetője erősíti meg pozíciójában, vagy jelöli ki a műhelytitkár javaslatára, majd a Műhely tagjai szavazással döntenek személyéről.
	\begin{enumerate}
		\item A Műhely vezetője megtagadhatja a javasolt személy jelölését; ez esetben új jelöltet kell állítani a műhelytitkári pozícióra.
		\item A Műhely tagjai abszolút többséggel választják meg az új műhelytitkárt.
	\end{enumerate}
	\item A műhelytitkár kinevezése rendesen egy évre szól.
	\begin{enumerate}
		\item A műhelytitkárt azonnali hatállyal felmentheti pozíciójából a Műhely vezetője, amennyiben megítélése szerint a rá bízott feladatait képtelen ellátni.
		\item A műhelytitkárnak joga van felmentését kérni a Műhely vezetőjétől, amennyiben javaslatot tesz számára a következő műhelytitkár személyére.
	\end{enumerate}
	\item A műhelytitkárt feladatai ellátásában állandó vagy egy adott feladat ellátására kijelölt
	személyek segíthetik. Feladataik koordinációjáért a műhelytitkár felel.
\end{enumerate}


\section{A vezető tanár}

\begin{enumerate}
	\item A Műhely vezető tanárának (a továbbiakban: vezető tanár) feladata a Műhely vezetőjének segítése, a Műhely és a műhelytitkár tevékenységei irányítása és koordinálása, valamint az aktív részvétel a Műhely tudományos programjain. A vezető tanár ezenfelül köteles félévente legalább egy műhelykurzust tartani.
	\item A vezető tanár megbízatásának keletkezése:
	\begin{enumerate}
		\item A vezető tanár a Műhely olyan végzett vagy jelenlegi tagja lehet, aki tanulmányait doktori képzésben végzi. Végzett tag esetén további feltétel a legalább négy félév aktív műhelytagság.
		\item A vezető tanárt a Műhely vezetője nevezi ki vagy erősíti meg pozíciójában.
	\end{enumerate}
	\item A vezető tanár megbízatásának megszűnése:
	\begin{enumerate}
		\item A vezető tanár megbízatása rendesen egy évre szól.
		\item A vezető tanárt a Műhely vezetője mentheti föl pozíciójából.
		\begin{enumerate}
			\item A vezető tanár kérelmezheti a Műhely vezetőjénél, hogy pozíciójából mentse föl.
		\end{enumerate}
	\end{enumerate}
	 \item A vezető tanár feladatait a Műhely vezetőjével való előzetes egyeztetés alapján végzi.
	 \item A vezető tanár jogosultságai megegyeznek a Műhely vezetőjének jelen szabályzatban részletezett jogosultságaival, azzal a kitétellel, hogy a vezető tanár a Műhely vezetőjének képviseletében, illetőleg egyetértésével gyakorolhatja e jogosultságokat.
	 \item Ha a Műhelynek nincs doktorandusz hallgatója, a Műhely vezetője nem nevez ki vezető tanárt, vagy a doktorandusz hallgató nem vállalja a tisztség betöltését, a pozíció betöltetlen marad.
\end{enumerate}


\section{Zárórendelkezések}

\begin{enumerate}
	\item Jelen szabályzat megfogalmazására a CTSzK 4. § (8) bekezdése alapján kapott felhatalmazást a Műhely.
	\item Jelen Szabályzat új műhelyszabályzat elfogadásáig érvényes.
	\begin{enumerate}
		\item Új műhelyszabályzat megalkotását a Műhely vezetője, a műhelytitkár, valamint a Műhely tagjai kezdeményezhetik. A műhelyszabályzat és módosítások elfogadására a félévnyitó műhelygyűlésen kerülhet sor, a Műhely vezetőjét, titkárát és tagjait is meghallgató érdemi vitát követően egyhangú szavazás útján.
		\item A Szabályzat elfogadására a CTSzK 4. § (8) bekezdése alapján, a műhelygyűlés javaslatára, a Collegium igazgatója jogosult.
	\end{enumerate}
	\item A műhelyszabályzatban nem szabályozott kérdések tekintetében a CTSzK vagy ELTE Hallgatói Követelményrendszer az irányadó.
	\begin{enumerate}
		\item A vitás kérdések elbírálására a Collegium igazgatója és a Műhely vezetője közötti megbeszélésen kerül sor.
	\end{enumerate}
\end{enumerate}

\end{document}