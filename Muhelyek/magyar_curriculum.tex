\documentclass{../styles/curriculum}

\begin{document}
\section*{ELTE Eötvös József Collegium \\ \vspace{0.5em} Curriculum \\ {\normalfont a Magyar műhely számára} } 

\vspace{2em}


\noindent A tanrend összeállításának alapelvei a következők voltak:
\begin{enumerate}
	\item A műhelykurzusok meghirdetésekor törekedni kell a műhelytagok érdeklődésének megfelelő sokszínűségre.
	\item A műhelykurzusok oktatói lehetőleg a Collegiumhoz köthető személyek.
	\item A műhelyszabályzat alapján a műhelytagok számára minden félévben legalább két kurzus elvégzése kötelező. Ebbe a Vitakör is beleértendő. Azon műhelykurzusok, amelyek egyetemi kódon kerülnek meghirdetésre, a két kötelező kurzusba csak annak a műhelytagnak számítanak bele, aki azt nem az egyetemi tanegységlista által előírt kódon, vagyis „szabad kreditben” végzi.
	\item A Műhely minden félévben – a Vitakörön és az egyetemi kurzusokat helyettesítő irodalmi szemináriumokon kívül – legalább két kurzust hirdet a műhelytagok számára.
\end{enumerate}
A Műhely kurzusai három csoportba sorolhatók:

\begin{enumerate}[label=\Roman*., itemsep=0ex]
	\item Minden műhelytag számára kötelező kurzus (Vitakör)
	\item Meghatározott félévben kötelezően elvégzendő szeminárium
	\item Félévenként változó tematikájú szeminárium.
\end{enumerate}


\section{Vitakör}

\begin{enumerate}
	\item A Vitakör a műhelymunka legfőbb terepe: a felsőbb éves műhelytagok egy előre meghatározott időpontban minden évben legalább egy dolgozatukat mutatják be. A kurzus legfőbb célja egymás munkáinak véleményezése, kritikai észrevételek megfogalmazása.
	\item A kurzus látogatása minden műhelytagnak kötelező.
	\item A Vitakört minden esetben a Műhelyhez szorosan kötődő egyetemi oktató vezeti.
	\item Indokolt esetben (pl. igen speciális témájú dolgozat esetén) a kurzust  vezető oktató mellett/helyett más, a témában jártas külső oktató is felkérhető a Vitakör levezetésére.
\end{enumerate}


\section{Meghatározott félévben kötelezően elvégzendő szemináriumok}

\begin{enumerate}[label=\arabic*.]
	\item \textbf{Irodalmi proszeminárium} (1. félév) \begin{enumerate}
		\item Az azonos című egyetemi kurzust helyettesítheti, célkitűzése szerint annál elmélyültebb ismereteket ad át.
		\item Az elsőéves műhelytagoknak kötelező kurzus.
	\end{enumerate}
	\item \textbf{Nyelvészeti  kutatásmódszertan} (1. vagy 2. félév)
	\begin{enumerate}
		\item Az azonos című egyetemi kurzust kiegészíti, annál elmélyültebb ismereteket ad át, de azt nem helyettesítheti.
		\item Az elsőéves műhelytagoknak kötelező kurzus.
	\end{enumerate}
	\item \textbf{Régi magyar irodalom; Klasszikus magyar irodalom; Modern magyar irodalom szemináriumok} (2–5. félév)
	\begin{enumerate}
		\item A Műhely legalább két félévenként meghirdeti a fenti szemináriumok mindegyikét, amelyek az azonos című egyetemi kurzust helyettesíthetik, azoknál specifikusabb képzést nyújtanak. A műhelytagok számára -- a mindenkori egyetemi tanegységlistának megfelelő félévben -- a Műhely által hirdetett kurzus elvégzése erősen ajánlott.
	\end{enumerate}	
	
\end{enumerate}


\section{Félévenként változó tematikájú szemináriumok}

\begin{enumerate}
	\item Az I. és II. pontban részletezett kurzusok mellett minden félévben legalább  két műhelyszeminárium kerül meghirdetésre.
	\item A műhelyszemináriumok oktatóiról és témájáról bármely műhelytag, a műhelyvezető vagy a kabinetvezető javaslata alapján a műhelygyűlés dönt.
	\item A kurzusok meghirdetésekor törekedni kell arra, hogy azok a teljes műhely érdeklődését lefedjék, annak megfelelően legyenek irodalomtudományi, nyelvtudományi és színháztudományi stb. tematikájú kurzusok.
	\item Törekedni kell a Collegium más műhelyeivel való együttműködésre, amelynek egyik pillére a közös műhelykurzusok meghirdetése.
	\item A kurzust az oktató egyedi elbírálása alapján más műhely tagja és nem collegista hallgató is felveheti.
	
\end{enumerate}

\vspace{2em}


Jelen tanrend a 2016. január 1-jén lép életbe.
\end{document}