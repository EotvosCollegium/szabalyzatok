\documentclass{../styles/rulebook}

\begin{document}
\section*{ELTE Eötvös József Collegium \\ Mendöl Tibor Földrajz--Földtudomány--Környezettudomány műhely\vspace{0.5em} Műhelyszabályzat} 

\vspace{2em}

\section*{Preambulum}
A régi Eötvös Collegium a magyar földrajz és a geológia oktatásának kiemelkedő műhelye volt. Ezen intézmény tagjai voltak a két világháború között iskolázódott geográfus nemzedék legnagyobb alakjai, többek között Mendöl Tibor, Bulla Béla, Kádár László, Wallner Ernő, Major Jenő, a geológus Szádecky-Kardoss Elemér és a mineralógus Mauritz Béla. A Collegium francia kötődésének is köszönhető, hogy a '20-as, '30-as évek magyar geográfiája (a korábban meghatározó német orientáció helyett) a francia emberföldrajz eszméi felé fordult; az emberföldrajzi iskola (mindenekelőtt Teleki Pál, Fodor Ferenc és Mendöl Tibor révén) a magyar geográfia egyik legnagyobb hatású irányzatává vált. A Collegium 1950-es megszüntetése a magyar földrajztudomány történetének is gyászos emlékű eseménye. Az 1957-ben újjászervezett Eötvös Kollégium a kezdeti időszakban kizárólag a bölcsészek előtt állt nyitva, és a hallgatók között később is csak elvétve akadt földrajz vagy geológus szakos. Érdemi változás az 1990-es évek végén következett be, amikor a Természettudományi Műhely keretei között újjászerveződött a földrajzi, földtudományi oktatás. 2002-ben alakult meg az önálló Földrajz-Földtudomány Műhely, amely a 2005-ös Mendöl centenáriumi évben felvette az egykori kitűnő kollégista nevét.


\section{Szabályok}

\begin{enumerate}
	\item Az ELTE Eötvös József Collegium (a továbbiakban Collegium) Mendöl Tibor Földrajz--Földtudomány--Környezettudomány műhelye (a továbbiakban műhely) a műhelytagok közössége, szakmai szervezete.
	\item A műhely tagja a Collegiumban főállásban vagy óraadóként földrajzi, földtudományi kurzusokat tartó tanár, valamint mindazon bentlakó vagy bejáró, kutató vagy tanár szakos collegista - a doktoranduszokat is beleértve -, akinek legalább egy szakja a földrajz-, ill. a földtudományokhoz kötődik (ilyen a geográfus, földrajz tanár, geológus, geofizikus, csillagász, környezettudomány, környezettan tanár szak, valamint a bolognai rendszerben a földrajz, földtudomány és környezettan alapszak, illetve az ezekre épülő mesterszakok).
	\item A fentieken túl a műhely tagja lehet minden collegista, aki felvételét kéri a műhelybe, és akinek felvételét a műhelygyűlés megszavazza.
	\item A műhely első embere a műhelyvezető. 
	\begin{enumerate}
		\item Megválasztásának körülményeit, illetve jogait és kötelességeit a Collegium Szervezeti és Működési Szabályzata (a továbbiakban SzMSz) ismerteti.
	\end{enumerate}
	\item A műhely évente egyszer műhelytitkárt választ.
	\begin{enumerate}
		\item A titkár feladata a műhelyvezető és a műhelytagság közti kapcsolattartás, a műhelyen belüli szervezési feladatok ellátása.
	\end{enumerate} 
	\item A műhely döntéshozatali fóruma a félévenként legalább egyszer sorra kerülő műhelygyűlés.
	\begin{enumerate}
		\item Ennek időpontjára a műhelyvezető tesz javaslatot, melyet - a titkár közreműködésével - előzetesen ismertet a műhely tagjaival.
		\item Bármely műhelytag kezdeményezésére rendkívüli műhelygyűlés hívható össze, ha ezt a műhely vezetője vagy a műhelytagok többsége támogatja.
		\item A műhelygyűlések -- a műhelytagok előzetes megállapodásától függően -- a Collegiumban vagy külső helyszínen is megtarthatók. 
		\item A műhelygyűléseken való részvétel minden műhelytag számára kötelező.
	\end{enumerate}
	\item A műhely az ötéves képzésben és a BSc oktatásban részt vevő hallgatók számára minden félévben lehetőség szerint legalább két (egy társadalomföldrajzi és egy földtudományi-természetföldrajzi) kurzust szervez. 
	\begin{enumerate}
		\item A későbbiekben az MSc képzésben részt vevő hallgatók számára további egy-egy kurzust hirdet meg, melyek teljesítése már egyéni kutatómunkát igényel, ugyanakkor tutori foglalkozással kiváltható.
		\item A még nem végzett műhelytagok számára legalább az egyik (lehetőleg az illető szakmai irányultságának megfelelő) kurzus rendszeres látogatása és teljesítése kötelező.
		\item  A kurzusokat más műhelyek tagjai és collegiumi tagsággal nem rendelkező egyetemi hallgatók is látogathatják.
	\end{enumerate} 
	\item A műhelytagok számára az MSc képzésben kötelező, a BSc képzésben másodévtől ajánlott a diákköri munkába való bekapcsolódás és a tudományos diákköri konferencián való részvétel, amit azonban kiválthat egy hazai vagy nemzetközi kiadványban megjelent szakirányú publikáció is.
	\item Az angol nyelvvizsgával nem rendelkező műhelytagok számára ajánlott az angol nyelv tanulása, valamint collegiumi tanulmányi éveik alatt legalább középfokú angol nyelvvizsga megszerzése.
	\item A műhely tevékenységének bemutatását a műhely honlapja szolgálja. 
	\begin{enumerate}
		\item A honlap aktualitásának fenntartása érdekében minden műhelytagnak el kell készítenie az oda felkerülő szakmai életrajzát, melynek tartalmát a későbbiekben évente legalább egyszer kötelező frissítenie.
	\end{enumerate}
	
\end{enumerate}


\section{Zárórendelkezések}

\begin{enumerate}
	\item Jelen szabályzat 2007. december 14-től lép életbe.
	\item Jelen szabályzat a Collegium Oktatási, Tanulmányi Szabályzatának és Követelményrendszerének (a továbbiakban CTSzK) mellékletét képezi.
	\item Minden egyéb kérdésben -- melyet jelen szabályzat nem részletez -- a CTSzK-t, valamint az SzMSz-t tekintjük mérvadónak.
\end{enumerate}

\end{document}