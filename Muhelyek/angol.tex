 \documentclass{../styles/rulebook}

\begin{document}
\section*{ELTE Eötvös József Collegium \\ Angol-Amerikai műhely\\ \vspace{0.5em} Műhelyszabályzat} 

\vspace{2em}

\section*{Preambulum}

Az Angol-Amerikai Műhely (a továbbiakban: Műhely) az Eötvös Loránd
Tudományegyetem (a továbbiakban: ELTE) Eötvös József Collegiumának (a
továbbiakban: Collegium) szakmai és hallgatói közössége. A Műhelyt a Collegium
Oktatási, Tanulmányi Szabályzata és Követelményrendszere (a továbbiakban:
CTSzK) 4. § (7) a. pontja hozza létre.

\section{Általános célkitűzések}

\begin{enumerate}
 \item A Műhely elsődleges célja az, hogy segítse az ELTE Bölcsészettudományi Kar
 legtehetségesebb hallgatóinak szakmai és tudományos előmenetelét. E cél elérésére
 elsődlegesen a CTSzK 4. § (3) bekezdése alapján meghirdethető műhelykurzusok,
 ezenfelül készségfejlesztő programok és személyre szabott tutorálás segítségével
 törekszik.
 \item A Műhely céljai továbbá
 \begin{enumerate}
	  \item konferencia- és publikálási lehetőség biztosítása a Műhely hallgatóinak;
	  \item a tevékenysége iránt érdeklődő középiskolás hallgatók megszólítása;
	  \item a Collegium és a Műhely népszerűsítése a magyarországi és határon túli magyar
	  nyelvű felsőoktatásban. 
 \end{enumerate}
\end{enumerate}

\section{Felvételi eljárás}

\begin{enumerate}	
	\item  A felvételi eljárás collegiumi szintű szabályairól a CTSzK 2. §-a rendelkezik.
	\item A Műhelybe az ELTE Bölcsészettudományi Karának anglisztika, angol tanári,
	valamint amerikanisztika hallgatói nyerhetnek felvételt.
	\item A Kar azon aktív jogviszonnyal rendelkező hallgatója jelentkezhet a Műhelybe, aki
	tanulmányait első- vagy másodéves alapképzésben vagy osztatlan tanárképzésben,
	illetve elsőéves mesterképzésben kezdi meg az adott felvételi időszak idején, a
	Műhely szempontjából releváns szakon.
	\begin{enumerate}
		\item A Műhely vezetőjének engedélyével az ELTE Bölcsészettudományi Kar más
		releváns szakjainak aktív jogviszonnyal rendelkező hallgatói is felvételt
		nyerhetnek.
		\item A Collegium igazgatójának és a Műhely vezetőjének egyetértésével az Egyetem
		más karain aktív jogviszonnyal rendelkező hallgató is felvételt nyerhet.
	\end{enumerate}
	\item  A Műhely szakmai felvételi bizottságának állandó tagjai a Műhely vezetője, valamint
	a műhelytitkár.
	\begin{enumerate}
		\item A Műhely vezetője a CTSzK 2. § (12) bekezdése értelmében jogosult
		távollétében jogait egy előzetesen megjelölt szaktanárra átruházni.
		\item A Műhely vezetője engedélyezheti a Műhely felsőbb éves tagjainak részvételét
		is a szakmai felvételi bizottságban.
		\item Ha a Műhelynek van vezetőtanára, ő is állandó tagja a szakmai felvételi
		bizottságnak. Távollétében a Műhely vezetője dönt az esetleges helyettesítéséről.
	\end{enumerate}	
	\item  A Műhelynek tiszteletbeli tagja lehet az ELTE BTK olyan, a Műhely szempontjából
	releváns szakokon aktív jogviszonnyal rendelkező hallgatója, aki tanulmányai során legalább három különböző félévben legalább három, a Műhelyben meghirdetett
	kurzuson részt vesz, és ott megfelelő érdemjegyet szerez. A tiszteletbeli műhelytag
	státusra a feltételeknek megfelelő hallgatót a Műhely vezetője vagy bármely tagja
	jelölheti, a státus odaítéléséről pedig a tagság a soron következő műhelygyűlésen
	egyszerű többséggel dönt. 
\end{enumerate}


\section{A Műhely tanulmányi rendje}

\begin{enumerate}
	\item A Műhely törekszik a hallgatók sokféleségének és érdeklődési körének megfelelő
	kurzuskínálat kialakítására.
	\item A Műhely ELTE BTK Angol-Amerikai Intézetében meghirdetett kurzusaira a
	Műhely tagjai, valamint az elérhető helyek függvényében az ELTE BTK anglisztika,
	angol tanári, amerikanisztika szakos hallgatói iratkozhatnak be. 
	\begin{enumerate}
		\item A Collegiumban
		meghirdetett kurzusokra kizárólag a Collegium hallgatói iratkozhatnak be.
		\item  Az adott
		műhelykurzus oktatója a Műhely vezetőjének engedélyével és a műhelytitkárral
		történt konzultálást követően, egyéni elbírálás alapján a Collegium más műhelyei
		tagjainak is engedélyezheti az óra felvételét. Ilyen esetben azt kell mérlegelni, hogy a	hallgató számára mennyire indokolt a kurzus felvétele, illetve megmarad-e az óra
		szemináriumi jellege.
	\end{enumerate}
	\item A Műhely rendes tagjának kötelességei az alábbiak.
	\begin{enumerate}
		\item A Műhely által szervezett kurzusok közül legalább egyet felvesz és teljesít minden félévben.
		\begin{enumerate}
			\item A műhelytitkár a Műhely vezetőjének egyetértése után ajánlást tesz, hogy a Műhely tagjainak mely kurzusokat érdemes az adott félévben elvégeznie.
			\item Az egyes kurzusok teljesítésének módjáról az óraadó tanárok döntenek, a	Műhely vezetőjének előzetes beleegyezésével.
			\item A Műhely tagja csak a Műhely vezetőjével, a műhelytitkárral és az adott kurzus
			oktatójával való előzetes konzultáció után, nyomós indok miatt adhatja le az
			órát, vagy dönthet annak fel nem vételéről.
			\item Műhelyórát tarthat a Műhely vezetőjének beleegyezésével a Műhely bármely
			tagja, amennyiben ezt számára az ELTE, illetve a Collegium szabályzata
			lehetővé teszi.
		\end{enumerate}
		\item Aktívan részt vesz a kari Tudományos Diákköri Konferencián (a továbbiakban: TDK) folyamatában, amellyel a Műhely igyekszik
		harmonizálni a műhelykonferenciák meghirdetését.
		\begin{enumerate}
			\item  Elsőéves hallgatóként
			figyelemmel kíséri a műhelykonferenciát, felsőbb éves hallgatóként a megadott
			határidőig előadástervet (absztraktot) nyújt be a meghirdetett
			műhelykonferenciára.
			\item Sikeres kutatómunka esetén jelentkezik a kétéves ciklusokban megrendezett Országos Tudományos Diákköri Konferenciára.
		\end{enumerate}
		\item Részt vesz a Műhely programjainak szervezésében.
		\item Szakmai előmeneteléről folyamatosan tájékoztatja a Műhely vezetőjét (elnyert díjak, pályázatok, versenyeredmények).
		\item Felsőbb éves hallgatóként segíti az újonnan felvett műhelytagokat szakmai
		fejlődésükben, a Collegiumba való beilleszkedésükben (tutoriális rendszer).
		\begin{enumerate}
			\item Minden újonnan felvett műhelytag lehetőség szerint az érdeklődésének megfelelő felsőbb éves hallgatót kap tutorként.
		\end{enumerate}
		\item A tavaszi félév vizsgaidőszakának végéig benyújt egy angol nyelvű irodalom,
		nyelvészet vagy kultúra témakörében írt, legalább 8 oldalas dolgozatot, amelyet a tutoron vagy a műhelytitkáron keresztül juttat el a Műhely vezetőjéhez. 
		\begin{enumerate}
			\item A dolgozat a Műhely vezetőjének előzetes engedélyével részben egy adott félévi
			szemináriumi dolgozat anyagára is épülhet, illetve az esetleges szakdolgozat tematikáját is követheti, valamint kapcsolatban állhat a 3. § (3) b. pontban taglalt TDK-kutatómunkával is.
		\end{enumerate}
		\item Ajánlott részt vennie a Műhely által szervezett tanulmányi verseny, illetve a tehetségtábor szervezésében, lebonyolításában.
		\item Ajánlott részt vennie a Műhely által a Kutatók Éjszakája rendezvénysorozatban
		nyújtott programok szervezésében, lebonyolításában.
		\item Ajánlott részt vennie a Collegiumban meghirdetett tudományos, kulturális és
		közösségi programokon, az Eötvös Konferencián, valamint a felvételi
		diákbizottságokban.
		\item A tiszteletbeli műhelytag nem köteles és nem is jogosult alanyi jogon a
		Collegium által meghirdetett nyelvórákat látogatni, nem jogosult a Collegium által
		meghirdetett kurzusokra beiratkozni, nem köteles féléves átlagáról beszámolni,
		nem illeti meg kollégiumi férőhely, továbbá a műhelygyűlésen csak tanácskozási
		jog illeti meg, de e korlátozásoktól és kivételektől eltekintve a rendes tagokkal
		azonos jogai vannak.
	\end{enumerate}
	\item A Műhely vezetőjének kötelességei a következők.
	\begin{enumerate}
		\item A műhelyvezetőt a Tanári Kar és a Műhelygyűlés javaslatára, a Kuratórium
		véleményének figyelembevételével az Igazgató bízza meg határozott időre, illetve menti fel, a Collegium Szervezeti és Működési
		Szabályzata (a továbbiakban: SzMSz) 31.§ (2) bekezdése alapján.
		\item Koordinálja a Műhely részvételét a tehetségtáborban, illetve a Kutatók Éjszakája
		eseménysorozatán.
		\item Felügyeli a Műhelyben a tutoriális munkát, illetve a hallgatók kutatómunkáját,	szakmai előmenetelét.
		\item Törekszik félévenként legalább egy olyan, törzsképzésen kívüli kurzus	megszervezésére, amelyet az ELTE BTK Angol-Amerikai Intézetének vagy
		valamely más felsőoktatási intézménynek a „külsős” tanára hirdet meg a	Collegiumban.
		\item Félévente meghirdet legalább egy szakos órát.
		\item A műhely tagjainak névsorát félévente frissíti, a műhelytitkár segítségével jegyzi a Műhely tagjai által elért ösztöndíjakat és eredményeket. 
	\end{enumerate}
\end{enumerate}


\section{A tagság megszűnése}

\begin{enumerate}
	\item A műhelytagság megszűnik a jelen Szabályzat 3. § (3) bekezdésében foglalt valamely
	feltétel vagy feltételek sorozatos nem teljesítése által.
	\begin{enumerate}
		\item E feltételek teljesítése alól a Műhely vezetőjének előzetes engedélye, illetőleg az
		utólag benyújtott méltányossági kérelem műhelyvezető általi kedvező elbírálása esetén mentesülhet a tag.
	\end{enumerate}
	\item A műhelytag jogviszonyának esetleges megszűnését a CTSzK 7. § (4)--(8) bekezdése szabályozza.
\end{enumerate}


\section{A műhelygyűlés}

\begin{enumerate}
	\item  A műhelygyűlés célja a Műhely vezetője, vezető tanára, titkára és tagjai közötti
	egyeztetésnek keretet adni.
	\item A műhelygyűlést a Műhely vezetője közvetlenül vagy a műhelytitkár közvetítésével
	hívja össze, elektronikus kommunikáció útján, a kijelölt időpontot legalább 8 nappal megelőzően.
	\item A műhelygyűlésen a műhely tagjainak joga van műhelytitkárt választani, munkáját
	véleményezni, illetve az Igazgató által kinevezett Műhelyvezető munkáját véleményezni, továbbá a Műhelyvezető személyére javaslatot tenni az SzMSz
	47. § (1) bekezdése alapján.
	\item Egy félévben minimálisan két (2) műhelygyűlés megtartására kerül sor.
	\begin{enumerate}
		\item A félévnyitó műhelygyűlésre a félév első vagy második hetében kerül sor. 
		\begin{enumerate}
			\item A félévnyitó műhelygyűlés célja
			az őszi félévben az újonnan felvett hallgatók megismerése, az előző félév
			munkájának összefoglalása, valamint a soron következő félév programjának
			megbeszélése.
			\item Az őszi félévnyitó műhelygyűlés során a Műhely webfejlesztőt jelöl és választ,
			aki a Műhely honlapjának és internetes megjelenéseinek frissítéséért felelős.
			\item Az őszi félévnyitó műhelygyűlésen kerül sor a műhelytitkár megválasztására is.
		\end{enumerate}
		\item A félévzáró műhelygyűlésre a téli szorgalmi időszak végén, illetőleg a tavaszi
		félév vizsgaidőszakában kerül sor. 
		\begin{enumerate}
			\item Célja a megelőző félév áttekintése, az
			elkövetkező programok felvázolása.
		\end{enumerate}
	\end{enumerate}
	\item A Műhely vezetője jogosult a műhelygyűlés időpontjának megváltoztatására.
	\item A műhelygyűlésen a Műhely minden tagja köteles részt venni.
	\begin{enumerate}
		\item A Műhely vezetője előzetes engedélyével, nyomós indokkal van mód a tagok
		műhelygyűléstől való távolmaradására.
		\item A műhelytitkár felel a műhelygyűlésen elhangzottak lejegyzéséért és annak a
		résztvevőkhöz történő eljuttatásáért. A jegyzőkönyv angol és/vagy magyar
		nyelven készülhet el.
		\item A műhelygyűlés határozatképes, amennyiben azon a Műhely tagságának
		legalább 50\%-a plusz egy fő megjelenik.
		\begin{enumerate}
			\item  Amennyiben a meghirdetett
			időpontban a műhelygyűlés határozatképtelen, a Műhely vezetője időbeli
			megkötés nélkül új időpontra hívhatja össze a műhelygyűlést, amely ezután már
			a részvételi aránytól függetlenül határozatképesnek minősül.
		\end{enumerate}
	\end{enumerate}
	\item A Műhely vezetője jogosult rendkívüli műhelygyűlés összehívására. 
\end{enumerate}


\section{A műhelytitkár}
\begin{enumerate}
	 \item A műhelytitkár a következő feladatokat látja el.
	 \begin{enumerate}
	 	\item  Kapcsolattartás a Műhely vezetője és tagjai, a Collegium tisztségviselői és
	 	szervei, valamint más a műhelyhez kötődő intézmények felé.
	 	\item  A Műhely képviselete a Collegium e célra szervezett fórumain, elsődlegesen a
	 	Tudományos Bizottság kiterjesztett ülésein.
	 	\item  A műhelygyűléseken jegyzőkönyv készítése, valamint e jegyzőkönyv továbbítása a Műhely vezetőjéhez és tagjaihoz.
	 	\item A Műhely rendezvényeinek és tevékenységeinek megszervezése, illetőleg a szervezés koordinációja.
	 	\item  A Műhely féléves beszámolójának megírása.
	 \end{enumerate}
	 \item A műhelytitkár a 6. § (1) pontban jelzett tevékenységeit a Műhely vezetője részéről
	 kapott utasítások alapján látja el. 
	 \begin{enumerate}
	 	\item A műhelytitkár folyamatosan tájékoztatja a Műhely
	 	vezetőjét az általa végzett tevékenységről.
	 	\item A Műhely egésze képviseletekor a műhelytitkár köteles előzetesen konzultálni a Műhely tagjaival is.
	 \end{enumerate}
	 \item A műhelytitkár a Műhely felsőbb éves hallgatója vagy a Műhely már végzett, de még aktív egyetemi hallgatói jogviszonnyal rendelkező hallgatója lehet.
	 \item A műhelytitkárt valamennyi őszi félévnyitó műhelygyűlésen a Műhely vezetője jelöli
	 ki, miután a Műhely tagjai titkos szavazással, egyszerű többséggel döntöttek a jelöltekről. 
	 \begin{enumerate}
	 	\item  A Műhely vezetője megtagadhatja a javasolt személy jelölését; ez esetben
	 	új jelöltet kell állítani a műhelytitkári pozícióra.
	 \end{enumerate}
	 \item A műhelytitkár kinevezése rendesen egy évre szól.
	 \begin{enumerate}
	 	\item A műhelytitkárt azonnali hatállyal felmentheti pozíciójából a Műhely vezetője,
	 	amennyiben megítélése szerint a rábízott feladatait képtelen ellátni.
	 	\item A műhelytitkárnak joga van felmentését kérni a Műhely vezetőjétől, amennyiben
	 	javaslatot tesz számára a következő műhelytitkár személyére.
	 \end{enumerate}
	 (6) A műhelytitkárt feladatai ellátásában állandó vagy egy adott feladat ellátására kijelölt
	 személyek segíthetik. Feladataik koordinációjáért a műhelytitkár felel. 
\end{enumerate}


\section{Együttműködés a Beloit College-dzsal}

\begin{enumerate}
	\item A Collegium a tavaszi félévben -- a Beloit College mindenkori döntésének függvényében -- egy hallgatót küldhet az amerikai Beloit College-ba.
	\begin{enumerate}
		\item Az ösztöndíjra jelentkezhet a Collegium minden olyan hallgatója, aki a kiutazást
		megelőző őszi félévben a Collegium aktív BA-, MA-, illetve osztatlan tanárképzésben résztvevő hallgatója,
		szakra és műhelyre tekintet nélkül.
		\item A jelentkezés határidejét és menetrendjét a Műhely vezetője az Igazgatóval egyeztetve határozza meg, de a jelentkezés határideje nem lehet
		később, mint november 20., az elbírálás pedig nem tarthat 8 naptári napnál tovább.
		\item Az ösztöndíj nem tartalmazza a kiutazás, a vízum-ügyintézés, a társadalombiztosítás, valamint az étkeztetés költségeit. 
		\begin{enumerate}
			\item A Műhely vezetője a
			Beloit College-dzsal kapcsolatos költségek, különösen az étkeztetés költségeit az
			együttműködés egyéb bevételeiből megpróbálja részben vagy egészben fedezni.
		\end{enumerate}
	\end{enumerate}
	\item Érdeklődés függvényében a Beloit College az őszi félévben vendéghallgató(ka)t küld
	a Collegiumba. 
	\begin{enumerate}
		\item A vendéghallgató(k) szállását a Collegium díjmentesen biztosítja, az oktatási és egyéb költségeket a vendéghallgató(k) részben önerőből, részben a Beloit
		College által e célra átutalt oktatási támogatásból fedezi(k).
		\item A vendéghallgató(ka)t tanulmányi és egyéb kérdésekben a Műhely vezetője, titkára,
		valamint esetleges önkéntesei segítik. 
		\item A tanrend kialakítása a Műhely vezetőjének
		tanácsai alapján történik.
	\end{enumerate}
\end{enumerate}


\section{Zárórendelkezések}

\begin{enumerate}
	\item Jelen szabályzat megfogalmazására a CTSzK 4. § (8) bekezdése alapján kapott
	felhatalmazást a Műhely.
	\item Jelen Szabályzat 2016. szeptember 1-jétől új műhelyszabályzat elfogadásáig
	érvényes. 
	\begin{enumerate}
		\item Új műhelyszabályzat megalkotását a Műhely vezetője, a műhelytitkár, valamint
		a Műhely tagjai kezdeményezhetik. A műhelyszabályzat és módosítások	elfogadására rendes vagy rendkívüli műhelygyűlés keretében kerülhet sor, a Műhely vezetőjét, titkárát és tagjait is meghallgató érdemi vitát követően
		egyhangú szavazás útján.
		\item A Szabályzatot a kiterjesztett Tudományos Bizottság véleményezi és jóváhagyja.
		\item A Szabályzat elfogadására a CTSzK 4. § (8) bekezdése alapján a Műhelygyűlés és a Tudományos Bizottság javaslatára a Collegium Igazgatója jogosult.
	\end{enumerate}
	\item A műhelyszabályzatban nem szabályozott kérdések tekintetében az SzMSz, a CTSzK vagy az ELTE
	Hallgatói Követelményrendszer az irányadó.
\end{enumerate}

\end{document}