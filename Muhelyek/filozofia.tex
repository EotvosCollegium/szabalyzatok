 \documentclass{../styles/rulebook}

\begin{document}
\section*{ELTE Eötvös József Collegium \\ Filozófia műhely\\ \vspace{0.5em} Műhelyszabályzat} 

\vspace{2em}

\section*{Preambulum}
TODO


\section{Bevezető}

\begin{enumerate}
	\item A Filozófia Műhely (a továbbiakban: Műhely) feladata segíteni tagjainak szakmai
	fejlődését az egyéni és a csoportos kutatómunka terén, valamint ellátni mind tagjai,
	mind a Műhely érdekképviseletét a megfelelő fórumokon.
	\item A Műhely feladatai közé tartozik a Műhely keretein belül tartott kurzusok
	megszervezése, a műhelytagok szakmai előmenetelének támogatása és figyelemmel
	kísérése.
	\item A Műhely feladatának tekinti továbbá, hogy időről időre, a Collegium többi
	műhelyével egyeztetve, más műhelyek tagjainak érdeklődésére számot tartó
	kurzusokat szervezzen, lehetőség szerint a kötelezően elvégzendő filozófia
	etalonkurzus keretében. Ezeken a kurzusokon a nem Eötvös-collegista hallgatók
	száma nem haladhatja meg az Eötvös-collegista hallgatók számának egyharmadát (33\%).
	\item A Műhely a Collegium Szervezeti és Működési Szabályzatának és Collegium Oktatási, Tanulmányi Szabályzatának és Követelményrendszerének (a továbbiakban CTSzK) megfelelően működik, a jelen műhelyszabályzatban leírt kiegészítésekkel.
\end{enumerate}


\section{A műhelytagság feltételei}

\begin{enumerate}
	\item  A Műhely rendes tagjai a Collegium filozófia szakos, illetve szabad bölcsész
	alapszak – filozófia szakirányos hallgatói.
	\item Eseti elbírálás alapján a műhelybe filozófia minoros, illetve más szakos
	hallgatók is felvételt nyerhetnek.
	\item A műhelytagságról való lemondás a műhelyvezetőhöz intézett írásbeli kérelem
	révén lehetséges, amelyet a következő műhelygyűlésen kell megvitatni.
	\begin{enumerate}
		\item  A műhelytagság megszűnésének feltétele a műhelygyűlés által hozott határozat.
	\end{enumerate}
\end{enumerate}


\section{A műhelygyűlés}

\begin{enumerate}
	\item A műhelygyűlés a Műhely legfőbb határozati szerve.
	\item A műhelygyűlés határozatképes, ha a rendes tagok legalább fele és a
	műhelyvezető (vagy felhatalmazottja) jelen van.
	\item A műhelygyűlés határozatait a
	jelenlévők többségi határozatával hozza.
	\item A műhelygyűlés hatókörébe tartozik többek között 
	\begin{enumerate}
		\item a Műhely jelen
		szabályzatának megváltoztatása; \item tisztségviselők választása; 
		\item a Műhely rövid- és hosszútávú működésének megszervezése.
	\end{enumerate}
	\item Műhelygyűlést félévente legalább kétszer, félév elején és végén kell tartani;
	továbbá minden olyan esetben, amikor műhelydolgozat megvitatására kerül sor.
	\item A félév eleji műhelygyűlés feladata megszervezni az adott félévet, megerősíteni
	vagy leváltani a tisztségviselőket, illetve megtervezni a következő félév menetét.
	\item A félév végi műhelygyűlés feladata az elmúlt félév értékelése, a tagok
	teljesítményének értékelése (különös tekintettel a Műhely által támasztott
	követelményekre), továbbá a következő félévi kurzusokra tett javaslatok
	megvitatása.
	\item A műhelytagok által benyújtott műhelydolgozatok megvitatására a tagok által
	közösen meghatározott időpontban kerül sor. Szükség esetén az időpontot a
	műhelyvezető is kitűzheti.
\end{enumerate}


\section{Tisztségek}

\begin{enumerate}
	\item A Műhely tisztségei: műhelyvezető, műhelytitkár, könyvtáros, pályázatfelelős.
	\begin{enumerate}
		\item Utóbbi hármat a rendes tagok közül választja a műhelygyűlés.	
	\end{enumerate}
	\item A műhelytitkár feladata a Műhely tevékenységének szervezése, a műhelyvezető
	munkájának segítése, a műhelykassza kezelése.
	\item A könyvtáros feladata kezelni, kölcsönözni és számon tartani a Műhely
	könyvtárának anyagát, és gondoskodni a rendszeres könyvadományok behajtásáról.
	A könyvtáros a műhelyvezetővel egyeztetve a könyvadományokra vonatkozó
	ajánlásokat tehet.
	\item A pályázatfelelős feladata a műhely munkájának anyagi feltételeit megteremtő
	pályázatok keresése, a pályázati anyag összeállítása, a pályázatok lebonyolítása.
\end{enumerate}


\section{A tagok kötelezettségei}

\begin{enumerate}
	\item A Műhely valamennyi tagja számára kötelező félévente két, a műhely által
	szervezett kurzus elvégzése, függetlenül attól, tagja-e másik műhelynek.
	\item A műhelyvezető a Műhely által szervezett valamennyi kurzushoz kurzusfelelőst
	rendelhet, akinek feladata a kurzus adminisztrálása, illetve a kurzussal kapcsolatos
	félév végi beszámoló megfogalmazása.
	\item A Műhely tagjai az általuk felvett, a Műhely által szervezett kurzusokról félévente
	legfeljebb három alkalommal hiányozhatnak. A hiányzások számának ellenőrzése a
	kurzusfelelős feladata.
	\item Indokolt esetben a műhelyvezető a műhely valamennyi tagja számára
	kötelezővé teheti a Műhely által szervezett valamely kurzus látogatását.
	\item Az egyetemen elsőéves hallgatók kivételével valamennyi tag kötelessége
	műhelydolgozat bemutatása. Műhelydolgozatot tanévenként legalább egyszer, a
	műhelyvezetővel és a többi taggal egyeztetett időpontban kell bemutatni.
	\item A műhelydolgozat témáját és rövid vázlatát a tanév kezdetétől számított 31
	napon belül valamennyi műhelytagnak írásban meg kell jelölnie, s a
	témamegjelölést a műhelyvezetőhöz és a műhelytitkárhoz írásban el kell juttatnia.
	\item A műhelydolgozat elkészítése során kötelező igénybe venni valamely, az
	egyetemen vagy a Műhelyben órát tartó oktató és / vagy felsőbbéves műhelytag
	segítségét, aki a műhelydolgozat megvitatásakor koreferensként szerepel.
	\item Valamennyi műhelytag kötelessége résztvenni valamennyi műhelydolgozat
	megvitatásán.
	\item Az egyetemen elsőéves hallgatók kivételével a Műhely valamennyi tagja köteles
	előadni az Eötvös Konferencián, és / vagy a Műhely által szervezett nyilvános
	konferencián.
	\item Valamennyi alapképzésben (BA) résztvevő hallgatónak kötelessége valamely,
	legalább mesterképzésben (MA) résztvevő műhelytaggal rendszeresen szakmai
	konzultációt tartania. 
	\begin{enumerate}
		\item A műhelytagot, az adott BA-s hallgató véleményének figyelembevételével, a műhelyvezető jelöli ki. 
		\item A kijelölt műhelytag az SzMSz által meghatározott felsőbb éves felelős jog- és feladatkörével rendelkezik.
	\end{enumerate}
	\item A műhelyvezető a Műhely bármely tagja számára kötelezővé teheti a következő kari
	TDK-n, továbbjutás esetén pedig az OTDK-n való részvételt, legalább egy naptári
	évvel az országos forduló megrendezése előtt.
	\item A Műhely tagjainak kötelessége a Műhely könyvtárát évente legalább egyszer
	könyvadománnyal támogatni.
	\begin{enumerate}
		\item Könyvadománynak minősülhet bármely, a Műhely profiljához kapcsolódó, nem fénymásolt és nem digitális formátumú elsődleges vagy másodlagos irodalom.
		\item A könyvtáros indokolt esetben visszautasíthatja a	könyvadományt, ez esetben a tag köteles új felajánlást tenni.
	\end{enumerate}
	\item A Műhely követelményeinek megszegése a műhelyből való kizárást vonhatja maga
	után. 
	\item A Műhely követelményei alól személyes és eseti elbírálás alapján a műhelyvezető
	adhat előzetes felmentést.
\end{enumerate}

\section{Zárórendelkezések}

\begin{enumerate}
	\item Jelen szabályzat XX-től hatályos.
	\item Jelen szabályzat a CTSzK mellékletét képezi.
\end{enumerate}

\end{document}